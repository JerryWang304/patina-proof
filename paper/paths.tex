\section*{L-values}

L-values in Patina are a combination of a variable and a path.
Paths are relative and specify subections of memory reachable from a L-value.

\[
\begin{array}{lccl}
\textrm{Variable} & x & & \\
\textrm{Path} & p & \bnfdef & \base \bnfalt \deref{p} \bnfalt \proj{p}{i} \bnfalt \unroll{\gt}{p} \\
\end{array}
\]

Projection ($\proj{p}{i}$) deconstructs tuples.
Unrolling ($\unroll{\gt}{p}$) deconstructs recursive types.
Dereference ($\deref{p}$) deconstructs pointers and references.
The base path ($\base$) does not deconstruct anything.

The typing judgment for paths does not present any surprises.
We use a partial map for the typing context and
the type substitution operation is the standard capture-avoiding substition.

$$ \gG : \mathrm{Variable} \to \mathrm{Type} $$

\fbox{$\tc{\gG}{\lv{x}{p}}{\gt}$}

\begin{mathpar}
\infer[PT-BASE]{\gG(x)=\gt}{\tc{\gG}{\lv{x}{\base}}{\gt}} \and
\infer[PT-DEOWN]{\tc{\gG}{\lv{x}{p}}{\own{\gt}}}{\tc{\gG}{\lv{x}{\deref{p}}}{\gt}} \and
\infer[PT-DEREF]{\tc{\gG}{\lv{x}{p}}{\tyref{\lt}{q}{\gt}}}{\tc{\gG}{\lv{x}{\deref{p}}}{\gt}} \and
\infer[PT-PROJ]{\tc{\gG}{\lv{x}{p}}{\subrec{\gt}{i}{n}}}{\tc{\gG}{\lv{x}{\proj{p}{i}}}{\gt_i}} \and
\infer[PT-UNROLL]
{\tc{\gG}{\lv{x}{p}}{\tyfix{X}{\gt}}}
{\tc{\gG}{\lv{x}{\unroll{\tyfix{X}{\gt}}{p}}}{\sub{X}{\tyfix{X}{\gt}}{\gt}}}
\end{mathpar}

Unsurprisingly, path typing is unique.
\begin{lem}[Path Type Uniqueness]
  If $\tc{\gG}{\lv{x}{p}}{\gt}$ and $\tc{\gG}{\lv{x}{p}}{\gt'}$ then $\gt = \gt' $.
\end{lem}

\begin{proof}[Path Type Uniqueness]
  We will use induction on the derivation of $\tc{\gG}{\lv{x}{p}}{\gt}$.
  \begin{itemize}
    \item[\textsc{PT-BASE}] Then $p = \base$.

      By inversion, $\gG(x)=\gt$.
      
      There is only one possible rule (\textsc{PT-BASE}) for 
      deriving $\tc{\gG}{\lv{x}{\base}}{\gt'}$ 
      
      Ergo, by inversion, $\gG(x)=\gt'$.

      Thus, $\gt = \gt'$.
    \item[\textsc{PT-DEOWN}] Then $p = \deref{p'}$.

      By inversion, $\tc{\gG}{\lv{x}{p'}}{\own{\gt}}$.

      There are two possible rules for deriving $\tc{\gG}{\lv{x}{\deref{p'}}}{\gt'}$.
      \begin{itemize}
	\item[\textsc{PT-DEOWN}]
	  By inversion, $\tc{\gG}{\lv{x}{p'}}{\own{\gt'}}$.

	  By induction $\own{\gt} =\ \own{\gt'}$.

	  Ergo $\gt = \gt'$.
	\item[\textsc{PT-DEREF}]
	  By inversion, $\tc{\gG}{\lv{x}{p'}}{\tyref{\lt}{q}{\gt'}}$.

	  By induction $\own{\gt} = \tyref{\lt}{q}{\gt'}$, a contradiction.

	  Thus, this case is impossible.
      \end{itemize}
    \item[\textsc{PT-DEREF}] Then $p = \deref{p'}$.

      Similar to the \textsc{PT-DEOWN} case, 
      but swapping the roles of \textsc{PT-DEOWN} and \textsc{PT-DEREF}.
    \item[\textsc{PT-PROJ}] Then $p = \proj{p'}{i}$ and $\gt = \gt_i$.

      By inversion, $\tc{\gG}{\lv{x}{p'}}{\subrec{\gt}{i}{n}}$.

      There is only one rule (\textsc{PT-PROJ}) for deriving $\tc{\gG}{\lv{x}{\proj{p'}{i}}}{\gt'}$.

      Thus, by inversion, $\gt' = \gt'_i$ and $\tc{\gG}{\lv{x}{p'}}{\subrec{\gt'}{i}{n'}}$.

      By induction, $\subrec{\gt}{i}{n} = \subrec{\gt'}{i}{n'}$.

      Ergo $n = n'$ and $\gt_i = \gt'_i$.
    \item[\textsc{PT-UNROLL}] 
      Then $p = \unroll{\tyfix{X}{\gt_0}}{p'}$ and $\gt = \sub{X}{\tyfix{X}{\gt_0}}{\gt_0}$.

      There is only one rule (\textsc{PT-UNROLL}) for deriving
      $\tc{\gG}{\lv{x}{\unroll{\tyfix{X}{\gt_0}}{p'}}}{\gt'}$.

      Thus, by inversion, $\gt' = \sub{X}{\tyfix{X}{\gt_0}}{\gt_0}$.

      Ergo $\gt = \sub{X}{\tyfix{X}{\gt_0}}{\gt_0} = \gt'$.
  \end{itemize}
\end{proof}

\section*{Runtime Memory}

\subsection*{Representation and Addressing}

To accurately model Rust's memory usage, Patina restricts
the contents of a memory cell to void data, an integer, or a pointer to another cell.
Tuples and recursive types have no physical memory presence (beyond contiguity for tuples).
The memory representation of a variant is a pair of a cell for the discriminant
and whatever memory is necessary to store the payload.

We could model this by a map from addresses to memory cell values,
but two issues make this inconvenient: the need for address arithmetic
and non-unique typing. However, we can add a little extra structure to
our model and eliminate these two issues. We wrap the cells inside
layouts describing the type structure overlaying memory.

We also separate plain pointer cells into owned cells and reference cells.
This is mostly useful for providing just enough type information about the pointer
for memory operation purposes. We will explain how we address memory in a moment,
so we use a placeholder for now.

\[
\begin{array}{lccl}
\textrm{Integer} & z & & \\
\textrm{Address} & \rho & & \textrm{a placeholder} \\
\textrm{Cell} & c & \bnfdef & \void \bnfalt z \bnfalt \own{\rho} \bnfalt \Ref{q~\rho} \\
\textrm{Layout} & l & \bnfdef & c \bnfalt \varval{c}{l} \bnfalt 
			       \subrec{l}{i}{n} \bnfalt \roll{\gt}{l} \\
\end{array}
\]

Due to the extra structure from layouts, addressing memory now requires
more than simple labels. Instead we use a runtime analogue of variables and paths.
Allocations are chunks of memory allocated and freed atomically.
They correspond to either variables or heap allocations.
Routes provide relative specification into layouts similar to paths with two key differences.

\[
\begin{array}{lccl}
\textrm{Allocation} & \ga & & \\
\textrm{Route} & r & \bnfdef & \base \bnfalt \proj{r}{i} \bnfalt \pay{r} \bnfalt \unroll{\gt}{r} \\
\end{array}
\]

The $\base$ route, projection, and unrolling are effectively identical to their path equivalents.
The $\pay{r}$ route refers to the payload $l$ in a variant $\varval{c}{l}$.
This is primarily used for match-by-reference.
Unlike paths, there is no dereference route.
This forces any pointer following into path evaluation (into routes)
rather than doing so while reading memory at a route.
The address placeholder from before is simply a pair of an allocation label and a route.
That is:

\[
\begin{array}{lccl}
\textrm{Integer} & z & & \\
\textrm{Allocation} & \ga & & \\
\textrm{Route} & r & \bnfdef & \base \bnfalt \proj{r}{i} \bnfalt \pay{r} \bnfalt \unroll{\gt}{r} \\
\textrm{Cell} & c & \bnfdef & \void \bnfalt z \bnfalt 
			      \own{\addr{\ga}{r}} \bnfalt \refval{q}{\addr{\ga}{r}} \\
\textrm{Layout} & l & \bnfdef & c \bnfalt \varval{c}{l} \bnfalt 
			       \subrec{l}{i}{n} \bnfalt \roll{\gt}{l} \\
\end{array}
\]

\subsection*{Reading}

Reading a layout from a route in memory is a straightforward operation.
We model the heap as a partial map from allocations to layouts.
Routes and layouts interact as you would expect.

$$ H : \mathrm{Allocation} \to \mathrm{Layout} $$

\fbox{$\Read{H}{\addr{\ga}{r}}{l}$}

\begin{mathpar}
\infer[RD-BASE]{H(\ga)=l}{\Read{H}{\addr{\ga}{\base}}{l}} \and
\infer[RD-PROJ]
  {\Read{H}{\addr{\ga}{r}}{\subrec{l}{i}{n}}}
  {\Read{H}{\addr{\ga}{\proj{r}{i}}}{l_i}} \and
\infer[RD-PAY]{\Read{H}{\addr{\ga}{r}}{\varval{c}{l}}}{\Read{H}{\addr{\ga}{\pay{r}}}{l}} \and
\infer[RD-UNROLL]{\Read{H}{\addr{\ga}{r}}{\roll{\gt'}{l}}}{\Read{H}{\addr{\ga}{\unroll{\gt}{r}}}{l}}
\end{mathpar}

As one would expect of a read operation, it's result is unique.

\begin{lem}[Read Uniqueness]
  If $\Read{H}{\addr{\ga}{r}}{l}$ and $\Read{H}{\addr{\ga}{r}}{l'}$, then $l = l'$.
\end{lem}

\begin{proof}[Read Uniqueness]
  We will induct on the derivation of $\Read{H}{\addr{\ga}{r}}{l}$.
  \begin{itemize}
    \item[\textsc{RD-BASE}] Then $r = \base$.

      By inversion, $H(\ga)=l$.

      There is only one rule (\textsc{RD-BASE}) for deriving $\Read{H}{\addr{\ga}{\base}}{l'}$.

      Thus, by inversion, $H(\ga)=l'$.

      Ergo, $l = l'$.
    \item[\textsc{RD-PROJ}] Then $r = \proj{r'}{i}$ and $l = l_i$.

      By inversion, $\Read{H}{\addr{\ga}{r'}}{\subrec{l}{i}{n}}$.

      There is only one rule (\textsc{RD-PROJ}) for
      deriving $\Read{H}{\addr{\ga}{\proj{r'}{i}}}{l'}$.

      Thus, by inversion, $\Read{H}{\addr{\ga}{r'}}{\subrec{l'}{i}{n'}}$.

      By induction, $\subrec{l}{i}{n} = \subrec{l'}{i}{n'}$.

      Ergo, $n = n'$ and $l_i = l'_i$.
    \item[\textsc{RD-PAY}] Then $r = \pay{r'}$.

      By inversion, $\Read{H}{\addr{\ga}{r'}}{\varval{c}{l}}$.

      There is only one rule (\textsc{RD-PAY}) for deriving $\Read{H}{\addr{\ga}{\pay{r'}}}{l'}$.

      Thus, by inversion, $\Read{H}{\addr{\ga}{r'}}{\varval{c'}{l'}}$.

      By induction, $\varval{c}{l} = \varval{c'}{l'}$.

      Ergo, $c = c'$ and $l = l'$.
    \item[\textsc{RD-UNROLL}] Then $r = \unroll{\gt}{r'}$.

      By inversion, $\Read{H}{\addr{\ga}{r'}}{\roll{\gt'}{l}}$.

      There is only one rule (\textsc{RD-UNROLL}) for
      deriving $\Read{H}{\addr{\ga}{\unroll{\gt}{r'}}}{l'}$.

      Thus, by inversion, $\Read{H}{\addr{\ga}{r'}}{\roll{\gt''}{l'}}$.

      By induction, $\roll{\gt'}{l} = \roll{\gt''}{l'}$.

      Ergo, $\gt' = \gt''$ and $l = l'$.
  \end{itemize}
\end{proof}

\section*{Path Evaluation}

Now that we can read memory, we can define path evaluation.
The only actual work of path evaluation is following the dereferences of pointers
to produce a route. In order to connect variables to runtime allocations, we use
a partial map tracking the allocation labels of variables.

$$ V : \mathrm{Variable} \to \mathrm{Allocation} $$

\fbox{$\ev{V;H}{\lv{x}{p}}{\addr{\ga}{r}}$}

\begin{mathpar}
\infer[PE-BASE]{V(x)=\ga}{\ev{V;H}{\lv{x}{\base}}{\addr{\ga}{\base}}} \and
\infer[PE-DEOWN]
  {\ev{V;H}{\lv{x}{p}}{\addr{\ga}{r}} \\ \Read{H}{\addr{\ga}{r}}{\own{\addr{\ga'}{r'}}}}
  {\ev{V;H}{\lv{x}{\deref{p}}}{\addr{\ga'}{r'}}} \and
\infer[PE-DEREF]
  {\ev{V;H}{\lv{x}{p}}{\addr{\ga}{r}} \\ \Read{H}{\addr{\ga}{r}}{\refval{q}{\addr{\ga'}{r'}}}}
  {\ev{V;H}{\lv{x}{\deref{p}}}{\addr{\ga'}{r'}}} \and
\infer[PE-PROJ]
  {\ev{V;H}{\lv{x}{p}}{\addr{\ga}{r}}}
  {\ev{V;H}{\lv{x}{\proj{p}{i}}}{\addr{\ga}{\proj{r}{i}}}} \and
\infer[PE-UNROLL]
  {\ev{V;H}{\lv{x}{p}}{\addr{\ga}{r}}}
  {\ev{V;H}{\lv{x}{\unroll{\gt}{p}}}{\addr{\ga}{\unroll{\gt}{r}}}}
\end{mathpar}

With path typing and evaluation now defined, we would like to prove progress
and preservation for them. Preservation will require we define typing for
runtime data (routes, cells, and layouts), which involves a slight bit of trickery
around variant discriminants and the payload route.

Progress will require a way to guarantee that the necessary \textsc{read}s 
can be completed, i.e. any pointers we need to dereference are in fact initialized
(a property we call \emph{shallow initialization}).
We will accomplish this by using a kind of shadow heap, which will simply track
initialization rather than values.
This will require an analogue of \textsc{read} that extracts the initialization
data for a path, a proof that this operation preserves types 
(which itselfs requires typing for this shadow heap),
a coherence check to ensure the shadow heap models the runtime heap,
and a proof that a shallowly initialized path evaluates to a readable route.

\section*{Runtime Typing}
\subsection*{Route Typing}
Typing routes is much like typing paths; however, the $\pay{r}$ route requires
special attention. Since this route points to the payload of a variant,
its type depends upon the value of the discriminant of the variant.
We handle this with an additional context that records the discriminant value
of a variant stored at a route.

$$ \gS : \mathrm{Allocation} \to \mathrm{Type}$$
$$ \gY : \mathrm{Allocation} \times \mathrm{Route} \to \mathrm{Integer} $$

\fbox{$\tc{\gS;\gY}{\addr{\ga}{r}}{\gt} $}

\begin{mathpar}
\infer[RT-BASE]{\gS(\ga)=\gt}{\tc{\gS;\gY}{\addr{\ga}{\base}}{\gt}}
\and
\infer[RT-PROJ]
  {\tc{\gS;\gY}{\addr{\ga}{r}}{\subrec{\gt}{i}{n}}}
  {\tc{\gS;\gY}{\addr{\ga}{\proj{r}{i}}}{\gt_i}}
\and
\infer[RT-PAY]
  {\gY(\ga,r)=i \\ \tc{\gS;\gY}{\addr{\ga}{r}}{\subvar{\gt}{i}{n}}}
  {\tc{\gS;\gY}{\addr{\ga}{\pay{r}}}{\gt_i}}
\and
\infer[RT-UNROLL]
  {\tc{\gS;\gY}{\addr{\ga}{r}}{\tyfix{X}{\gt}}}
  {\tc{\gS;\gY}{\addr{\ga}{\unroll{\tyfix{X}{\gt}}{r}}}{\sub{X}{\tyfix{X}{\gt}}{\gt}}}
\end{mathpar}

\subsection*{Layout and Cell Typing}
Typing for layouts and cells is what you would expect.
However, we restrict the type of $\void$ to the possible types for other
kinds of cells, i.e. $\tyint$, unique pointers, and borrowed references.
This means that multi-cell layout types (variants, tuples, and recursive types) imply
a non-$\void$ value.
\newline

\fbox{$\tc{\gS;\gY}{l}{\gt}$}

\begin{mathpar}
\infer[LT-VOIDINT]
{ }
{\tc{\gS;\gY}{\void}{\tyint}}
\and
\infer[LT-VOIDOWN]
{ }
{\tc{\gS;\gY}{\void}{\own{\gt}}}
\and
\infer[LT-VOIDREF]
{ }
{\tc{\gS;\gY}{\void}{\tyref{\lt}{q}{\gt}}}
\and
\infer[LT-INT]
{ }
{\tc{\gS;\gY}{z}{\tyint}}
\and
\infer[LT-OWN]
{\tc{\gS;\gY}{\addr{\ga}{r}}{\gt}}
{\tc{\gS;\gY}{\own{\addr{\ga}{r}}}{\own{\gt}}}
\and
\infer[LT-REF]
{\tc{\gS;\gY}{\addr{\ga}{r}}{\gt}}
{\tc{\gS;\gY}{\refval{q}{\addr{\ga}{r}}}{\tyref{\lt}{q}{\gt}}}
\and
\infer[LT-VOIDVAR]
{ }
{\tc{\gS;\gY}{\varval{\void}{\void}}{\subvar{\gt}{i}{n}}}
\and
\infer[LT-VAR]
{\tc{\gS;\gY}{l}{\gt_i}}
{\tc{\gS;\gY}{\varval{i}{l}}{\subvar{\gt}{i}{n}}}
\and
\infer[LT-REC]
{\forall i.~\tc{\gS;\gY}{l_i}{\gt_i}}
{\tc{\gS;\gY}{\subrec{l}{i}{n}}{\subrec{\gt}{i}{n}}}
\and
\infer[LT-ROLL]
{\tc{\gS;\gY}{l}{\sub{X}{\tyfix{X}{\gt}}{\gt}}}
{\tc{\gS;\gY}{\roll{\tyfix{X}{\gt}}{l}}{\tyfix{X}{\gt}}}
\end{mathpar}

\subsection*{Heap Well-Formedness}
With runtime typing now in hand, we can say what it means for the heap to be well-formed.
We say a heap $H$ is described by a heap type $\gS$ and a discriminant context $\gY$ if
every allocation in $H$ has the type assigned to it by $\gS$
and $\gY$ correctly records the discrimiants of all the variants in $H$.
Formally,

\begin{mathpar}
\mprset{flushleft}
\infer
{ 
\dom{H} = \dom{\gS} \\\\
\forall \ga \in \dom(\gS).~\gS(\ga)~\textrm{closed} \\\\
\forall \ga \in \dom{H}.~\tc{\gS;\gY}{H(\ga)}{\gS(\ga)} \\\\
\forall (\ga,r) \in \dom{\gY}.~\Read{H}{\addr{\ga}{r}}{\varval{\gY(\ga,r)}{l}}
}
{\tc{}{H}{\gS;\gY}}
\end{mathpar}

\section*{Read Safety}
We are now in position to prove that our read operation is safe.
That is, if the heap is well formed and a route is has some type,
then we can read some layout at that route and that layout has that same type.

\begin{lem}[Read Safety]
If $\tc{}{H}{\gS;\gY}$ and $\tc{\gS;\gY}{\addr{\ga}{r}}{\gt}$,
then there is a layout $l$ such that $\Read{H}{\addr{\ga}{r}}{l}$ and $\tc{\gS;\gY}{l}{\gt}$.
\end{lem}

\begin{proof}[Read Safety]
  We will use induction on the derivation of $\tc{\gS;\gY}{\addr{\ga}{r}}{\gt}$.
  \begin{itemize}
    \item[\textsc{RT-BASE}] Then $r = \base$.

      By inversion, $\gS(\ga)=\gt$.

      From $\tc{}{H}{\gS;\gY}$, we know that $\dom{H}=\dom{\gS}$.
      Thus, $\exists l.~H(\ga)=l$.

      Then by \textsc{RD-BASE}, we have $\Read{H}{\addr{\ga}{\base}}{l}$.

      Again from $\tc{}{H}{\gS;\gY}$, we know that $\tc{\gS;\gY}{H(\ga)}{\gS(\ga)}$,
      which is just $\tc{\gS;\gY}{l}{\gt}$ as required.
    \item[\textsc{RT-PROJ}] Then $r = \proj{r'}{i}$ and $\gt = \gt_i$.

      By inversion, $\tc{\gS;\gY}{\addr{\ga}{r'}}{\subrec{\gt}{i}{n}}$.

      By induction, $\Read{H}{\addr{\ga}{r'}}{l}$
      and $\tc{\gS;\gY}{l}{\subrec{\gt}{i}{n}}$ for some $l$.

      There is only one rule (\textsc{LT-REC}) for deriving $\tc{\gS;\gY}{l}{\subrec{\gt}{i}{n}}$.

      Thus, by inversion, $l = \subrec{l'}{j}{n}$ 
      and $\forall j.~\tc{\gS;\gY}{l'_j}{\gt_j}$.
      Specifically, $\tc{\gS;\gY}{l'_i}{\gt_i}$.

      Then by \textsc{RD-PROJ}, we have $\Read{H}{\addr{\ga}{\proj{r'}{i}}}{l'_i}$.
    \item[\textsc{RT-PAY}] Then $r = \pay{r'}$ and $\gt = \gt_i$.

      By inversion, $\gY(\ga,r')=i$ and $\tc{\gS;\gY}{\addr{\ga}{r'}}{\subvar{\gt}{i}{n}}$.

      By induction, $\Read{H}{\addr{\ga}{r'}}{l}$
      and $\tc{\gS;\gY}{l}{\subvar{\gt}{i}{n}}$.

      From $\tc{}{H}{\gS;\gY}$, we know that because $\gY(\ga,r')=i$
      it follows that $\Read{H}{\addr{\ga}{r'}}{\varval{i}{l'}}$ for some $l'$.

      By Read Uniqueness, we know that $l = \varval{i}{l'}$.

      There is only one rule (\textsc{LT-VAR}) for deriving 
      $\tc{\gS;\gY}{\varval{i}{l'}}{\subvar{\gt}{i}{n}}$.

      Thus, by inversion, $\tc{\gS;\gY}{l'}{\gt_i}$.

      Then by \textsc{RD-PAY}, we have $\Read{H}{\addr{\ga}{\pay{r'}}}{l'}$.
    \item[\textsc{RT-UNROLL}] 
      Then $r = \unroll{\tyfix{X}{\gt'}}{r'}$ and $\gt = \sub{X}{\tyfix{X}{\gt'}}{\gt'}$.

      By inversion, $\tc{\gS;\gY}{\addr{\ga}{r'}}{\tyfix{X}{\gt'}}$.

      By induction, $\Read{H}{\addr{\ga}{r'}}{l}$ 
      and $\tc{\gS;\gY}{l}{\tyfix{X}{\gt'}}$ for some $l$.

      There is only one rule (\textsc{LT-ROLL}) for deriving $\tc{\gS;\gY}{l}{\tyfix{X}{\gt'}}$.

      Thus, by inversion, $l = \roll{\tyfix{X}{\gt'}}{l'}$
      and $\tc{\gS;\gY}{l'}{\sub{X}{\tyfix{X}{\gt'}}{\gt'}}$.

      Then by \textsc{RD-UNROLL}, we have
      $\Read{H}{\addr{\ga}{\unroll{\tyfix{X}{\gt'}}{r'}}}{l'}$.
  \end{itemize}
\end{proof}

\section*{Path Type Preservation}
The final piece we need for preservation is how to relate the typing context $\gG$,
the variable map $V$, and the heap type $\gS$. This is simply the requirement that
the map $V$ preserves typing.

\begin{mathpar}
  \infer
  {\forall x \in \dom{\gG}.~\gG(x)=\gS(V(x))}
  {\gG;V\vdash\gS}
\end{mathpar}

Finally, we can state and prove path type preservation.
Under well-formed heap and contexts, typed paths evaluate to routes of the same type.

\begin{lem}[Path Type Preservation]
  If $\tc{}{H}{\gS;\gY}$, $\gG;V\vdash\gS$, $\tc{\gG}{\lv{x}{p}}{\gt}$,
  and $\ev{V;H}{\lv{x}{p}}{\addr{\ga}{r}}$, then $\tc{\gS;\gY}{\addr{\ga}{r}}{\gt}$.
\end{lem}

\begin{proof}[Path Type Preservation]
  We will use induction on the derivation of $\ev{V;H}{\lv{x}{p}}{\addr{\ga}{r}}$.
  \begin{itemize}
    \item[PE-BASE] Then $p = \base$ and $r = \base$.

      By inversion, $V(x)=\ga$.

      There is only one rule (\textsc{PT-BASE}) for deriving $\tc{\gG}{\lv{x}{\base}}{\gt}$.

      Thus, by inversion, we have $\gG(x)=\gt$.

      From $\gG;V\vdash\gS$, we know $\gG(x)=\gS(V(x))$. Thus, $\gt=\gS(\ga)$.

      Then by \textsc{RT-BASE}, we have $\tc{\gS;\gY}{\addr{\ga}{\base}}{\gt}$.
    \item[PE-DEOWN] Then $p = \deref{p'}$.

      By inversion, $\ev{V;H}{\lv{x}{p'}}{\addr{\ga'}{r'}}$
      and $\Read{H}{\addr{\ga'}{r'}}{\own{\addr{\ga}{r}}}$.

      There are two possible rules for deriving $\tc{\gG}{\lv{x}{\deref{p'}}}{\gt}$.
      \begin{itemize}
	\item[PT-DEOWN]

	  By inversion, $\tc{\gG}{\lv{x}{p'}}{\own{\gt}}$.

	  By induction, $\tc{\gS;\gY}{\addr{\ga'}{r'}}{\own{\gt}}$.

	  By Read Safety, we have $\Read{H}{\addr{\ga'}{r'}}{l}$
	  and $\tc{\gS;\gY}{l}{\own{\gt}}$ for some $l$.

	  By Read Uniqueness, we have $l =\ \own{\addr{\ga}{r}}$.

	  There is only one rule (\textsc{LT-OWN}) for deriving 
	  $\tc{\gS;\gY}{\own{\addr{\ga}{r}}}{\own{\gt}}$.

	  Thus, by inversion, $\tc{\gS;\gY}{\addr{\ga}{r}}{\gt}$.
	\item[PT-DEREF]

	  By inversion, $\tc{\gG}{\lv{x}{p'}}{\tyref{\lt}{q}{\gt}}$.

	  By induction, $\tc{\gS;\gY}{\addr{\ga'}{r'}}{\tyref{\lt}{q}{\gt}}$.

	  By Read Safety, we have $\Read{H}{\addr{\ga'}{r'}}{l}$
	  and $\tc{\gS;\gY}{l}{\tyref{\lt}{q}{\gt}}$ for some $l$.

	  By Read Uniqueness, we have $l =\ \own{\addr{\ga}{r}}$.

	  There are no rules for deriving
	  $\tc{\gS;\gY}{\own{\addr{\ga}{r}}}{\tyref{\lt}{q}{\gt}}$.

	  Thus, by inversion, this case is impossible.
      \end{itemize}
    \item[PE-DEREF] 
      Similar to the \textsc{PE-DEOWN} case, but switching the roles of
      \textsc{PT-DEOWN} and \textsc{PT-DEREF}.
    \item[PE-PROJ] Then $p = \proj{p'}{i}$ and $r = \proj{r'}{i}$.

      By inversion, $\ev{V;H}{\lv{x}{p'}}{\addr{\ga}{r'}}$.

      There is only one rule (\textsc{PT-PROJ}) for deriving $\tc{\gG}{\lv{x}{\proj{p'}{i}}}{\gt}$.

      Thus, by inversion, we have $\gt = \gt_i$ and $\tc{\gG}{\lv{x}{p'}}{\subrec{\gt}{i}{n}}$.

      By induction, we have $\tc{\gS;\gY}{\addr{\ga}{r'}}{\subrec{\gt}{i}{n}}$.

      Then by \textsc{RT-PROJ}, we have $\tc{\gS;\gY}{\addr{\ga}{\proj{r'}{i}}}{\gt_i}$.
    \item[PE-UNROLL] Then $p = \unroll{\gt'}{p'}$ and $r = \unroll{\gt'}{r'}$.

      By inversion, $\ev{V;H}{\lv{x}{p'}}{\addr{\ga}{r'}}$.

      There is only one rule (\textsc{PT-UNROLL}) for deriving
      $\tc{\gG}{\lv{x}{\unroll{\gt'}{p'}}}{\gt}$.

      Thus, by inversion, we have $\gt' = \tyfix{X}{\gt''}$,
      $\gt = \sub{X}{\tyfix{X}{\gt''}}{\gt''}$, and
      $\tc{\gG}{\lv{x}{p'}}{\tyfix{X}{\gt''}}$.

      By induction, we have $\tc{\gS;\gY}{\addr{\ga}{r'}}{\tyfix{X}{\gt''}}$.

      Then by \textsc{RT-UNROLL}, we have
      $\tc{\gS;\gY}{\addr{\ga}{\unroll{\tyfix{X}{\gt''}}{r'}}}{\sub{X}{\tyfix{X}{\gt''}}{\gt''}}$.
  \end{itemize}
\end{proof}
