\documentclass{article}
\usepackage{mathpartir}
\usepackage{fullpage}
\usepackage{amsmath}
\usepackage{amsthm}
\usepackage{amsfonts}
\usepackage{mathrsfs}

% domain shorthand
\DeclareMathOperator{\dom}{dom}
\begin{document}

\newtheorem*{lem}{Lemma}
\newtheorem*{conj}{Conjecture}

% BNF shorthand
\newcommand \bnfdef {::=}
\newcommand \bnfalt {~|~}

% symbol shorthand
\newcommand \own[1] {\sim{#1}}
\newcommand \Ref[1] {\&~{#1}}

\newcommand \var[1] {\langle{#1}\rangle}
\newcommand \subvar[3] {\var{{#1}_{#2}}_{{#2} \in \{1 \ldots {#3}\} }}

\newcommand \rec[1] {[{#1}]}
\newcommand \subrec[3] {\rec{{#1}_{#2}}_{{#2} \in \{1 \ldots {#3}\} }}

% lifetime shorthand
\newcommand \lt {\ell}
\newcommand \ltr {\mathscr{L}}

% Greek shorthands
\newcommand \ga {\alpha}
\newcommand \gG {\Gamma}
\newcommand \gs {\sigma}
\newcommand \gS {\Sigma}
\newcommand \gt {\tau}
\newcommand \gU {\Upsilon}
\newcommand \gY {\Psi}

% qual shorthand
\newcommand \qimm {\mathsf{imm}}
\newcommand \qmut {\mathsf{mut}}
\newcommand \mqnone {\mathsf{none}}

% type shorthand
\newcommand \tyint {\mathsf{int}}
\newcommand \tyref[3] {\Ref{{#1}~{#2}~{#3}}}
\newcommand \tyfix[2] {\mu {#1}.{#2}}

% path/route shorthand
\newcommand \lv[2] {{#1}@{#2}}
\newcommand \addr[2] {{#1}@{#2}}
\newcommand \base {\bullet}
\newcommand \deref[1] {*{#1}}
\newcommand \pay[1] {\mathsf{pay}~{#1}}
\newcommand \proj[2] {{#1}\cdot{#2}}
\newcommand \unroll[2] {\mathsf{unroll}~[{#1}]~{#2}}

% cell shorthand
\newcommand \void {\mathsf{void}}
\newcommand \refval[2] {\Ref{{#1}~{#2}}}
\newcommand \varval[2] {\var{{#1}:{#2}}}
\newcommand \roll[2] {\mathsf{roll}~[{#1}]~{#2}}

% hole shorthand
\newcommand \uninit {\mathsf{uninit}}

% statement shorthand
\newcommand \nop {\mathsf{skip}}
\newcommand \seq[2] {{#1};{#2}}
\newcommand \set[2] {{#1} \leftarrow {#2}}
\newcommand \new[2] {{#1} \leftarrow \textsf{new}\ {#2}}
\newcommand \swap[2] {{#1} \leftrightarrow {#2}}
\newcommand \free[1] {\textsf{free}\ {#1}}
\newcommand \push[4] {\textsf{push}\ {#1}\ {#2}\ {#3}\ {#4}}
\newcommand \pop[2] {\textsf{pop}\ {#1}\ {#2}}

% judgment shorthand
\newcommand \tc[3] {{#1} \vdash {#2} : {#3}}
\newcommand \sub[3] {[{#1} \mapsto {#2}] {#3}}
\newcommand \Read[3] {\textsc{read}({#1},{#2},{#3})}
\newcommand \ev[3] {{#1} \vdash {#2} \to {#3}}
\newcommand \shallow[3] {\textsc{shallow}({#1},{#2},{#3})}
\newcommand \initShadow[1] {\textsc{init}({#1})}
\newcommand \bankwf[3] {{#1} \vdash {#2} \leq {#3}}
\newcommand \tcb[5] {{#1} \vdash {#2} : {#3}\ \textrm{controlled by}\ {#4} \leq {#5}}
\newcommand \tb[4] {{#1} \vdash {#2} : {#3} \leq {#4}}
\newcommand \nm[1] {\textsc{not-mut}({#1})}
\newcommand \canread[2] {\textsc{can-read}({#1},{#2})}
\newcommand \cnml[1] {\textsc{contains-no-mut-loans}({#1})}
\newcommand \cw[2] {\textsc{can-write}({#1},{#2})}
\newcommand \cnl[1] {\textsc{contains-no-loans}({#1})}
\newcommand \cm[2] {\textsc{can-move}({#1},{#2})}
\newcommand \tycopy[1] {{#1}\ \textrm{copy}}
\newcommand \tymove[1] {{#1}\ \textrm{move}}
\newcommand \uselv[5] {{#1};{#2}\vdash{#3}:{#4}\Rightarrow{#5}}
\newcommand \valid[5] {{#1};{#2};{#3}\vdash{#4}\ \textrm{valid for}\ {#5}}
\newcommand \freezable[4] {{#1};{#2}\vdash{#3}\ \textrm{freezable for}\ {#4}}
\newcommand \unique[4] {{#1};{#2}\vdash{#3}\ \textrm{unique for}\ {#4}}

\section*{Abstract}
Rust is a new systems language that uses some advanced type system features,
specifically affine types and regions, to statically guarantee memory
safety and eliminate the need for a garbage collector.
While each individual addition to the type system is well understood
in isolation and are known to be sound, the combined system is not.
Furthermore, Rust uses a novel checking scheme for its regions, known
as the Borrow Checker, that is not known to be correct.
Since Rust's goal is to be a safer alternative to C/C++, we should
ensure that this safety scheme actually works. We present a formal
semantics that captures the key features relevant to memory safety,
unique pointers and borrowed references, specifies how they guarantee
memory safety, and describes the operation of the Borrow Checker.
We use this model to prove the soudness of some core operations
and justify the conjecture that the model, as a whole, is sound.
Additionally, our model provides a syntactic version of the Borrow
Checker, which may be more understandable than the non-syntactic version
in Rust.

\section*{Overview}
The Rust Programming Language~\cite{rust} makes strong claims about 
ensuring memory safety without garbage collection.
We would like to prove that those claims are true.
To that end, we use a small model of Rust, called Patina, that characterizes the key
features of Rust under the most suspicion, namely its memory management.

Rust memory management has two goals:
no memory should ever become unreachable without being deallocated
and no uninitialized memory should ever be read.
Rust tries to achieve this by maintaining two invariants:
all memory has a unique owner responsible for deallocating it and
no memory is ever simultaneously aliased and mutable.
These invariants simplify the situation enough so that
Rust only needs to track which L-values are uninitialized and which have been borrowed.
Ownership tracking of heap memory is managed by unique pointers.
Safe aliasing is managed by borrowed references.

The Patina model has three layers.
The innermost layer deals with L-values and ensures 
no uninitialized data is ever used, which includes ensuring
that uninitialized pointers and null pointers are never dereferenced.
It also ensures that compile-time initialization information correctly
models the runtime memory.
The middle layer deals with R-values.
It ensures that any L-values used do not violate initialization or borrow restrictions.
It also ensures that borrowed references are not created unless they are safe.
The outer layer deals with statements.
It mostly just chains together the guarantees of the L-value and R-value layers.
However, it is also responsible for ensuring no leaks would be created when deallocating memory.

\subsection*{Unique Pointers in Rust}
In Rust, unique pointers are the owners of heap memory.
Heap memory is allocated when a new unique pointer is created.
Heap memory is freed when a unique pointer falls out of scope.
\begin{verbatim}
{
  let x: ~int; // A stack variable containing a unique pointer to an integer
  x = ~3; // Allocates a heap integer, initializes it with 3, and stores the pointer in x
  // The heap memory owned by x is freed when it falls out of scope
}
\end{verbatim}

To avoid double frees, unique pointers must remain unique.
Thus, when a unqiue pointer would be copied the original must be rendered unsuable.
That is, they are moved rather than copied.
\begin{verbatim}
{
  let x: ~int = ~3;
  let y: ~int = x; // x is moved into y. x is no longer usable.
  let z: ~int = x; // ERROR!
}
\end{verbatim}

However, these deinitialized paths can be reinitialized.
\begin{verbatim}
{
  let x: ~int = ~3;
  let y: ~int = x; // x is now deinitialized
  x = ~1; // x is initialized again
}
\end{verbatim}

Unique pointers can also be freed if they would be leaked by assignment.
\begin{verbatim}
{
  let x: ~~int = ~~3;
  *x = ~1; // The ~3 that *x points too would be leaked here. It is freed instead.
}
\end{verbatim}

This is not a dynamic check. 
The compiler detects when heap memory should be freed and inserts the necessary code.

\subsection*{Unique Pointers in Patina}
When unique pointers fall out of scope in Rust, the compiler inserts code at the end
of the block that will actually free the allocated heap memory. This code will
traverse the entire owned data structure and free memory without leaking anything.
In the case of recursive data structures, this code will be a recursive function.
Rather than model such complex behavior as a primitive in Patina, we will use only
a shallow free statement. Aside from making the model simpler, this also provides us
with a means of checking that the code Rust inserts to free memory is correct,
i.e.\ by encoding it in terms of Patina's shallow frees.
The only way to deallocate a unique pointer in Patina is with these explicit frees.
\begin{verbatim}
{
  let x: ~~int = ~~3;
  //free(x); // ERROR! would leak *x
  // The proper way
  free(*x);
  free(x);
}
\end{verbatim}
However, the free statement is only valid for initialized pointers, which prevents double frees.
\begin{verbatim}
{
  let x: ~int = ~3;
  free(x); // *x is now deallocated. x is uninitialized
  free(x); // ERROR! x is not initialized
}
\end{verbatim}
Finally, the free statement requires that the pointer is unaliased, preventing dangling pointers.
\begin{verbatim}
{
  let x: ~int = ~3;
  // Create an immutable borrowed reference to *x with &
  // Immutable borrowed references to some type A are denoted by &A
  let y: &int = &*x;
  free(x); // ERROR! would make y a dangling pointer
}
\end{verbatim}

\subsection*{Borrowed References}
Often, particularly in functions, we want to use some memory, but do not intend to free it.
That is, we want to borrow access to memory owned by something else.
This is fine for copyable values, like integers, 
but is tedious for noncopyable values such as unique pointers.
To avoid freeing a unique pointer argument at the end of the function body,
it must be returned from the function.
Effectively, the programmer must thread noncopyable values through their function calls.
\begin{verbatim}
fn foo(x: ~int) -> (int, ~int) {
  return (*x + 1, x)
}

{
  let x1: ~int = ~1;
  let (y,x2): (int, ~int) = foo(x);
  // y = 2, x1 is no longer valid, x2 = ~1
}
\end{verbatim}

Borrowed references allow programmers to have temporary access to a value,
but they do not confer ownership so this tedium is not necessary.
\begin{verbatim}
fn foo(x: &int) -> int {
  return (*x + 1)
}

{
  let x: ~int = ~1;
  let y: int = foo(&*x); // Rust lets us write foo(x) here and will insert the &* for us
  // x = ~1, y = 2
}
\end{verbatim}

However, unrestricted borrowed references would open a hole in our memory safety
by enabling aliased, mutable memory.
\begin{verbatim}
{
  let x: ~int = ~1;
  let y: &~int = &x; // x and *y are the same memory location
  let z: ~int = x; // x is no longer usable - effectively x is uninitialized memory
  let w: int = **y; // ERROR! *y is uninitialized now, so **y dereferences a null pointer!
}

{
  let x: &int;
  {
    let y: int = 3;
    y = &x;
  } // x would be freed here
  let z: int = *x // ERROR *y would point to unallocated memory
}

{
  let x: Option<int> = Some(3);
  let y: &int = match x {
    Some(ref z) => z,
    None => fail!(), // Impossible
  }
  x = None // payload is now uninitialized
  let z: int = *y; // ERROR! *y points to uninitialized memory
}
\end{verbatim}

Aliased, mutable memory can be avoided in two ways: forbidding mutability or forbidding aliasing.
Correspondingly, Rust has two kinds of borrowed references.
Immutable borrows, denoted \texttt{\&}, are aliasable,
but require all memory reachable through them to be immutable for the duration of the borrow.
\begin{verbatim}
{
  let x: ~int = ~1;
  let y: &~int = &x; // this immutable borrow prevents x from being changed
  let z: ~int = x; // Rust will not let this typecheck because it would change x 
}
\end{verbatim}

Conversely, mutable borrows, denoted \texttt{\&mut}, are mutable,
but require all memory reachable through them to be \emph{only} accessible through the borrow
for the duration. That is, they guarantee unique access.
\begin{verbatim}
{
  let x: ~int = ~1;
  let y: &mut ~int = &mut x; // this mutable borrow prevents others from using x
  let z: ~int = x; // Rust will not let this typecheck because it would access x
}
\end{verbatim}

Finally, both kinds of borrowed reference require that the referent outlive the reference,
which prevents dangling references.
\begin{verbatim}
fn foo() -> &int {
  let x: int = 1;
  return &x; // Rust will not let this typecheck because it would be a dangling pointer
}
\end{verbatim}

\section*{Types}
Values in Patina can be described by the following types:
\[
\begin{array}{lccl}
\textrm{Lifetime} & \lt & & \\
\textrm{Type Variable} & X & & \\
\textrm{Qualifier} & q & \bnfdef & \qimm \bnfalt \qmut \\
\textrm{Type} & \gt & \bnfdef & \tyint \bnfalt \own{\gt} \bnfalt \tyref{\lt}{q}{\gt} \bnfalt 
				\subvar{\gt}{i}{n} \bnfalt \subrec{\gt}{i}{n} \bnfalt 
				\tyfix{X}{\gt} \bnfalt X \\
\end{array}
\]

$\own{\gt}$ is a unique pointer to a $\gt$.
$\tyref{\lt}{q}{\gt}$ is a borrowed reference to a $\gt$
providing mutability guarantee $q$ for lifetime $\lt$, which is Rust's name
for regions associated with stack frames.
$\subvar{\gt}{i}{n}$ is a discriminated union or a variant type.
$\subrec{\gt}{i}{n}$ is a tuple type.
Finally, $\tyfix{X}{\gt}$ is a recursive type.

\section*{L-values}

L-values in Patina are a combination of a variable and a path.
Paths are relative and specify subections of memory reachable from a L-value.
Projection ($\proj{p}{i}$) deconstructs tuples.
Unrolling ($\unroll{\gt}{p}$) deconstructs recursive types.
Dereference ($\deref{p}$) deconstructs pointers and references.
The base path ($\base$) does not deconstruct anything.

\[
\begin{array}{lccl}
\textrm{Variable} & x & & \\
\textrm{Path} & p & \bnfdef & \base \bnfalt \deref{p} \bnfalt \proj{p}{i} \bnfalt \unroll{\gt}{p} \\
\end{array}
\]

The L-value layer of Patina has two core operations: evaluating a path
to a memory location and reading the value at that location.
We need to characterize the circumstances under which these operations will progress,
and we need to show that these operations preserve certain information, specifically
type and initialization data. Higher layers of Patina will build on these properties.
We shall start with typing paths and then move into defining evaluation for paths.

The typing judgment for paths does not present any surprises.
We use a partial map for the typing context and
the type substitution operation is the standard capture-avoiding substition.

$$ \gG : \mathrm{Variable} \to \mathrm{Type} $$

\fbox{$\tc{\gG}{\lv{x}{p}}{\gt}$}

\begin{mathpar}
\infer[PT-BASE]{\gG(x)=\gt}{\tc{\gG}{\lv{x}{\base}}{\gt}} \and
\infer[PT-DEOWN]{\tc{\gG}{\lv{x}{p}}{\own{\gt}}}{\tc{\gG}{\lv{x}{\deref{p}}}{\gt}} \and
\infer[PT-DEREF]{\tc{\gG}{\lv{x}{p}}{\tyref{\lt}{q}{\gt}}}{\tc{\gG}{\lv{x}{\deref{p}}}{\gt}} \and
\infer[PT-PROJ]{\tc{\gG}{\lv{x}{p}}{\subrec{\gt}{i}{n}}}{\tc{\gG}{\lv{x}{\proj{p}{i}}}{\gt_i}} \and
\infer[PT-UNROLL]
{\tc{\gG}{\lv{x}{p}}{\tyfix{X}{\gt}}}
{\tc{\gG}{\lv{x}{\unroll{\tyfix{X}{\gt}}{p}}}{\sub{X}{\tyfix{X}{\gt}}{\gt}}}
\end{mathpar}

\section*{Runtime Memory}

\subsection*{Representation and Addressing}

To accurately model Rust's memory usage, Patina restricts
the contents of a memory cell to void data, an integer, or a pointer to another cell.
Tuples and recursive types have no physical memory presence (beyond contiguity for tuples).
The memory representation of a variant is a pair of a cell for the discriminant
and whatever memory is necessary to store the payload.

We could model this by a map from addresses to memory cell values,
but two issues make this inconvenient: the need for address arithmetic
and non-unique typing. However, we can add a little extra structure to
our model and eliminate these two issues. We wrap the cells inside
layouts describing the type structure overlaying memory.

We also separate plain pointer cells into owned cells and reference cells.
This is mostly useful for providing just enough type information about the pointer
for memory operation purposes. We will explain how we address memory in a moment,
so we use a placeholder for now.

\[
\begin{array}{lccl}
\textrm{Integer} & z & & \\
\textrm{Address} & \rho & & \textrm{a placeholder} \\
\textrm{Cell} & c & \bnfdef & \void \bnfalt z \bnfalt \own{\rho} \bnfalt \Ref{q~\rho} \\
\textrm{Layout} & l & \bnfdef & c \bnfalt \varval{c}{l} \bnfalt 
			       \subrec{l}{i}{n} \bnfalt \roll{\gt}{l} \\
\end{array}
\]

Due to the extra structure from layouts, addressing memory now requires
more than simple labels. Instead we use a runtime analogue of variables and paths.
Allocations are chunks of memory allocated and freed atomically.
They correspond to either variables or heap allocations.
Routes provide relative specification into layouts similar to paths with two key differences.

\[
\begin{array}{lccl}
\textrm{Allocation} & \ga & & \\
\textrm{Route} & r & \bnfdef & \base \bnfalt \proj{r}{i} \bnfalt \pay{r} \bnfalt \unroll{\gt}{r} \\
\end{array}
\]

The $\base$ route, projection, and unrolling are effectively identical to their path equivalents.
The $\pay{r}$ route refers to the payload $l$ in a variant $\varval{c}{l}$.
This is primarily used for match-by-reference.
Unlike paths, there is no dereference route.
This forces any pointer following into path evaluation (into routes)
rather than doing so while reading memory at a route.
The address placeholder from before is simply a pair of an allocation label and a route.
The exception is that unique pointers needs only an allocation since a unique pointer
always points to a complete heap allocation.
That is:

\[
\begin{array}{lccl}
\textrm{Integer} & z & & \\
\textrm{Allocation} & \ga & & \\
\textrm{Route} & r & \bnfdef & \base \bnfalt \proj{r}{i} \bnfalt \pay{r} \bnfalt \unroll{\gt}{r} \\
\textrm{Cell} & c & \bnfdef & \void \bnfalt z \bnfalt 
			      \own{\ga} \bnfalt \refval{q}{\addr{\ga}{r}} \\
\textrm{Layout} & l & \bnfdef & c \bnfalt \varval{c}{l} \bnfalt 
			       \subrec{l}{i}{n} \bnfalt \roll{\gt}{l} \\
\end{array}
\]

\subsection*{Reading}

Reading a layout from a route in memory is a straightforward operation.
We model the heap as a partial map from allocations to layouts.
Routes and layouts interact as you would expect.

$$ H : \mathrm{Allocation} \to \mathrm{Layout} $$

\fbox{$\Read{H}{\addr{\ga}{r}}{l}$}

\begin{mathpar}
\infer[RD-BASE]{H(\ga)=l}{\Read{H}{\addr{\ga}{\base}}{l}} \and
\infer[RD-PROJ]
  {\Read{H}{\addr{\ga}{r}}{\subrec{l}{i}{n}}}
  {\Read{H}{\addr{\ga}{\proj{r}{i}}}{l_i}} \and
\infer[RD-PAY]{\Read{H}{\addr{\ga}{r}}{\varval{c}{l}}}{\Read{H}{\addr{\ga}{\pay{r}}}{l}} \and
\infer[RD-UNROLL]{\Read{H}{\addr{\ga}{r}}{\roll{\gt'}{l}}}{\Read{H}{\addr{\ga}{\unroll{\gt}{r}}}{l}}
\end{mathpar}

As one would expect of a read operation, it's result is unique.

\begin{lem}[Read Uniqueness]
  If $\Read{H}{\addr{\ga}{r}}{l}$ and $\Read{H}{\addr{\ga}{r}}{l'}$, then $l = l'$.
\end{lem}

\begin{proof}[Read Uniqueness]
  We will induct on the derivation of $\Read{H}{\addr{\ga}{r}}{l}$.
  \begin{itemize}
    \item[\textsc{RD-BASE}] Then $r = \base$. \\
      By inversion, $H(\ga)=l$. \\
      There is only one rule (\textsc{RD-BASE}) for deriving $\Read{H}{\addr{\ga}{\base}}{l'}$. \\
      Thus, by inversion, $H(\ga)=l'$. \\
      Ergo, $l = l'$.
    \item[\textsc{RD-PROJ}] Then $r = \proj{r'}{i}$ and $l = l_i$. \\
      By inversion, $\Read{H}{\addr{\ga}{r'}}{\subrec{l}{i}{n}}$. \\
      There is only one rule (\textsc{RD-PROJ}) for
      deriving $\Read{H}{\addr{\ga}{\proj{r'}{i}}}{l'}$. \\
      Thus, by inversion, $\Read{H}{\addr{\ga}{r'}}{\subrec{l'}{i}{n'}}$. \\
      By induction, $\subrec{l}{i}{n} = \subrec{l'}{i}{n'}$. \\
      Ergo, $n = n'$ and $l_i = l'_i$.
    \item[\textsc{RD-PAY}] Then $r = \pay{r'}$. \\
      By inversion, $\Read{H}{\addr{\ga}{r'}}{\varval{c}{l}}$. \\
      There is only one rule (\textsc{RD-PAY}) for deriving $\Read{H}{\addr{\ga}{\pay{r'}}}{l'}$. \\
      Thus, by inversion, $\Read{H}{\addr{\ga}{r'}}{\varval{c'}{l'}}$. \\
      By induction, $\varval{c}{l} = \varval{c'}{l'}$. \\
      Ergo, $c = c'$ and $l = l'$.
    \item[\textsc{RD-UNROLL}] Then $r = \unroll{\gt}{r'}$. \\
      By inversion, $\Read{H}{\addr{\ga}{r'}}{\roll{\gt'}{l}}$. \\
      There is only one rule (\textsc{RD-UNROLL}) for
      deriving $\Read{H}{\addr{\ga}{\unroll{\gt}{r'}}}{l'}$. \\
      Thus, by inversion, $\Read{H}{\addr{\ga}{r'}}{\roll{\gt''}{l'}}$. \\
      By induction, $\roll{\gt'}{l} = \roll{\gt''}{l'}$. \\
      Ergo, $\gt' = \gt''$ and $l = l'$.
  \end{itemize}
\end{proof}

\section*{Path Evaluation}

Now that we can read memory, we can define path evaluation.
The only actual work of path evaluation is following the dereferences of pointers
to produce a route. In order to connect variables to runtime allocations, we use
a partial map tracking the allocation labels of variables.

$$ V : \mathrm{Variable} \to \mathrm{Allocation} $$

\fbox{$\ev{V;H}{\lv{x}{p}}{\addr{\ga}{r}}$}

\begin{mathpar}
\infer[PE-BASE]{V(x)=\ga}{\ev{V;H}{\lv{x}{\base}}{\addr{\ga}{\base}}} \and
\infer[PE-DEOWN]
  {\ev{V;H}{\lv{x}{p}}{\addr{\ga}{r}} \\ \Read{H}{\addr{\ga}{r}}{\own{\ga'}}}
  {\ev{V;H}{\lv{x}{\deref{p}}}{\addr{\ga'}{\base}}} \and
\infer[PE-DEREF]
  {\ev{V;H}{\lv{x}{p}}{\addr{\ga}{r}} \\ \Read{H}{\addr{\ga}{r}}{\refval{q}{\addr{\ga'}{r'}}}}
  {\ev{V;H}{\lv{x}{\deref{p}}}{\addr{\ga'}{r'}}} \and
\infer[PE-PROJ]
  {\ev{V;H}{\lv{x}{p}}{\addr{\ga}{r}}}
  {\ev{V;H}{\lv{x}{\proj{p}{i}}}{\addr{\ga}{\proj{r}{i}}}} \and
\infer[PE-UNROLL]
  {\ev{V;H}{\lv{x}{p}}{\addr{\ga}{r}}}
  {\ev{V;H}{\lv{x}{\unroll{\gt}{p}}}{\addr{\ga}{\unroll{\gt}{r}}}}
\end{mathpar}

Our core operations are now defined,
but we still need to describe the typing our runtime data
so that we can properly state and prove our type preservation lemmas.

\section*{Runtime Typing}
\subsection*{Route Typing}
Typing routes is much like typing paths; however, the $\pay{r}$ route requires
special attention. Since this route points to the payload of a variant,
its type depends upon the value of the discriminant of the variant.
This is used for handling match by reference.
The match arms have a pointer to the payload in scope, but the type of the
payload depends upon the value of the discrimiant.
For example,
\begin{verbatim}
  // x: <[], int> is a Patina encoding of Option<int>
  match x by imm y // y will have route `pay r' were `r' is the route of x
    // [] (None) case
    {
      // y: & imm [] is in scope here
    }
    // int (Some) case
    {
      // y: & imm int is in scope here
    }
\end{verbatim}

We handle this with an additional context that records the discriminant value
of a variant stored at a route. 

$$ \gS : \mathrm{Allocation} \to \mathrm{Type}$$
$$ \gY : \mathrm{Allocation} \times \mathrm{Route} \to \mathrm{Integer} $$

\fbox{$\tc{\gS;\gY}{\addr{\ga}{r}}{\gt} $}

\begin{mathpar}
\infer[RT-BASE]{\gS(\ga)=\gt}{\tc{\gS;\gY}{\addr{\ga}{\base}}{\gt}}
\and
\infer[RT-PROJ]
  {\tc{\gS;\gY}{\addr{\ga}{r}}{\subrec{\gt}{i}{n}}}
  {\tc{\gS;\gY}{\addr{\ga}{\proj{r}{i}}}{\gt_i}}
\and
\infer[RT-PAY]
  {\gY(\ga,r)=i \\ \tc{\gS;\gY}{\addr{\ga}{r}}{\subvar{\gt}{i}{n}}}
  {\tc{\gS;\gY}{\addr{\ga}{\pay{r}}}{\gt_i}}
\and
\infer[RT-UNROLL]
  {\tc{\gS;\gY}{\addr{\ga}{r}}{\tyfix{X}{\gt}}}
  {\tc{\gS;\gY}{\addr{\ga}{\unroll{\tyfix{X}{\gt}}{r}}}{\sub{X}{\tyfix{X}{\gt}}{\gt}}}
\end{mathpar}

\subsection*{Layout and Cell Typing}
Typing for layouts and cells is what you would expect.
However, we restrict the type of $\void$ to the possible types for other
kinds of cells, i.e. $\tyint$, unique pointers, and borrowed references.
This means that multi-cell layout types (variants, tuples, and recursive types) imply
a non-$\void$ value.
\newline

\fbox{$\tc{\gS;\gY}{l}{\gt}$}

\begin{mathpar}
\infer[LT-VOIDINT]
{ }
{\tc{\gS;\gY}{\void}{\tyint}}
\and
\infer[LT-VOIDOWN]
{ }
{\tc{\gS;\gY}{\void}{\own{\gt}}}
\and
\infer[LT-VOIDREF]
{ }
{\tc{\gS;\gY}{\void}{\tyref{\lt}{q}{\gt}}}
\and
\infer[LT-INT]
{ }
{\tc{\gS;\gY}{z}{\tyint}}
\and
\infer[LT-OWN]
{\tc{\gS;\gY}{\addr{\ga}{\base}}{\gt}}
{\tc{\gS;\gY}{\own{\ga}}{\own{\gt}}}
\and
\infer[LT-REF]
{\tc{\gS;\gY}{\addr{\ga}{r}}{\gt}}
{\tc{\gS;\gY}{\refval{q}{\addr{\ga}{r}}}{\tyref{\lt}{q}{\gt}}}
\and
\infer[LT-VOIDVAR]
{ }
{\tc{\gS;\gY}{\varval{\void}{\void}}{\subvar{\gt}{i}{n}}}
\and
\infer[LT-VAR]
{\tc{\gS;\gY}{l}{\gt_i}}
{\tc{\gS;\gY}{\varval{i}{l}}{\subvar{\gt}{i}{n}}}
\and
\infer[LT-REC]
{\forall i.~\tc{\gS;\gY}{l_i}{\gt_i}}
{\tc{\gS;\gY}{\subrec{l}{i}{n}}{\subrec{\gt}{i}{n}}}
\and
\infer[LT-ROLL]
{\tc{\gS;\gY}{l}{\sub{X}{\tyfix{X}{\gt}}{\gt}}}
{\tc{\gS;\gY}{\roll{\tyfix{X}{\gt}}{l}}{\tyfix{X}{\gt}}}
\end{mathpar}

\subsection*{Heap Well-Formedness}
With runtime typing now in hand, we can say what it means for the heap to be well-formed.
We say a heap $H$ is described by a heap type $\gS$ and a discriminant context $\gY$ if
every allocation in $H$ has the type assigned to it by $\gS$
and $\gY$ correctly records the discrimiants of all the variants in $H$.
Formally,

\begin{mathpar}
\mprset{flushleft}
\infer
{ 
\dom{H} = \dom{\gS} \\\\
\forall \ga \in \dom(\gS).~\gS(\ga)~\textrm{closed} \\\\
\forall \ga \in \dom{H}.~\tc{\gS;\gY}{H(\ga)}{\gS(\ga)} \\\\
\forall (\ga,r) \in \dom{\gY}.~\Read{H}{\addr{\ga}{r}}{\varval{\gY(\ga,r)}{l}}
}
{\tc{}{H}{\gS;\gY}}
\end{mathpar}

\section*{Read Safety}
We are now in position to prove that our read operation is safe.
That is, if the heap is well formed and a route is has some type,
then we can read some layout at that route and that layout has that same type.
This is effectively a merger of the type preservation and progress lemmas for \textsc{read}.

\begin{lem}[Read Safety]
If $\tc{}{H}{\gS;\gY}$ and $\tc{\gS;\gY}{\addr{\ga}{r}}{\gt}$,
then there is a layout $l$ such that $\Read{H}{\addr{\ga}{r}}{l}$ and $\tc{\gS;\gY}{l}{\gt}$.
\end{lem}

\begin{proof}[Read Safety]
  We will use induction on the derivation of $\tc{\gS;\gY}{\addr{\ga}{r}}{\gt}$.
  \begin{itemize}
    \item[\textsc{RT-BASE}] Then $r = \base$. \\
      By inversion, $\gS(\ga)=\gt$. \\
      From $\tc{}{H}{\gS;\gY}$, we know that $\dom{H}=\dom{\gS}$.
      Thus, $\exists l.~H(\ga)=l$. \\
      Then by \textsc{RD-BASE}, we have $\Read{H}{\addr{\ga}{\base}}{l}$. \\
      Again from $\tc{}{H}{\gS;\gY}$, we know that $\tc{\gS;\gY}{H(\ga)}{\gS(\ga)}$,
      which is just $\tc{\gS;\gY}{l}{\gt}$ as required.
    \item[\textsc{RT-PROJ}] Then $r = \proj{r'}{i}$ and $\gt = \gt_i$. \\
      By inversion, $\tc{\gS;\gY}{\addr{\ga}{r'}}{\subrec{\gt}{i}{n}}$. \\
      By induction, $\Read{H}{\addr{\ga}{r'}}{l}$
      and $\tc{\gS;\gY}{l}{\subrec{\gt}{i}{n}}$ for some $l$. \\
      There is only one rule (\textsc{LT-REC}) for deriving $\tc{\gS;\gY}{l}{\subrec{\gt}{i}{n}}$. \\
      Thus, by inversion, $l = \subrec{l'}{j}{n}$ 
      and $\forall j.~\tc{\gS;\gY}{l'_j}{\gt_j}$.
      Specifically, $\tc{\gS;\gY}{l'_i}{\gt_i}$. \\
      Then by \textsc{RD-PROJ}, we have $\Read{H}{\addr{\ga}{\proj{r'}{i}}}{l'_i}$.
    \item[\textsc{RT-PAY}] Then $r = \pay{r'}$ and $\gt = \gt_i$. \\
      By inversion, $\gY(\ga,r')=i$ and $\tc{\gS;\gY}{\addr{\ga}{r'}}{\subvar{\gt}{i}{n}}$. \\
      By induction, $\Read{H}{\addr{\ga}{r'}}{l}$
      and $\tc{\gS;\gY}{l}{\subvar{\gt}{i}{n}}$. \\
      From $\tc{}{H}{\gS;\gY}$, we know that because $\gY(\ga,r')=i$
      it follows that $\Read{H}{\addr{\ga}{r'}}{\varval{i}{l'}}$ for some $l'$. \\
      By Read Uniqueness, we know that $l = \varval{i}{l'}$. \\
      There is only one rule (\textsc{LT-VAR}) for deriving 
      $\tc{\gS;\gY}{\varval{i}{l'}}{\subvar{\gt}{i}{n}}$. \\
      Thus, by inversion, $\tc{\gS;\gY}{l'}{\gt_i}$. \\
      Then by \textsc{RD-PAY}, we have $\Read{H}{\addr{\ga}{\pay{r'}}}{l'}$.
    \item[\textsc{RT-UNROLL}] 
      Then $r = \unroll{\tyfix{X}{\gt'}}{r'}$ and $\gt = \sub{X}{\tyfix{X}{\gt'}}{\gt'}$. \\
      By inversion, $\tc{\gS;\gY}{\addr{\ga}{r'}}{\tyfix{X}{\gt'}}$. \\
      By induction, $\Read{H}{\addr{\ga}{r'}}{l}$ 
      and $\tc{\gS;\gY}{l}{\tyfix{X}{\gt'}}$ for some $l$. \\
      There is only one rule (\textsc{LT-ROLL}) for deriving $\tc{\gS;\gY}{l}{\tyfix{X}{\gt'}}$. \\
      Thus, by inversion, $l = \roll{\tyfix{X}{\gt'}}{l'}$
      and $\tc{\gS;\gY}{l'}{\sub{X}{\tyfix{X}{\gt'}}{\gt'}}$. \\
      Then by \textsc{RD-UNROLL}, we have
      $\Read{H}{\addr{\ga}{\unroll{\tyfix{X}{\gt'}}{r'}}}{l'}$.
  \end{itemize}
\end{proof}

\section*{Path Type Preservation}
The final piece we need for preservation is how to relate the typing context $\gG$,
the variable map $V$, and the heap type $\gS$. This is simply the requirement that
the map $V$ preserves typing.

\begin{mathpar}
  \infer
  {\forall x \in \dom{\gG}.~\gG(x)=\gS(V(x))}
  {\gG;V\vdash\gS}
\end{mathpar}

Finally, we can state and prove path type preservation.
Under well-formed heap and contexts, typed paths evaluate to routes of the same type.

\begin{lem}[Path Type Preservation]
  If $\tc{}{H}{\gS;\gY}$, $\gG;V\vdash\gS$, $\tc{\gG}{\lv{x}{p}}{\gt}$,
  and $\ev{V;H}{\lv{x}{p}}{\addr{\ga}{r}}$, then $\tc{\gS;\gY}{\addr{\ga}{r}}{\gt}$.
\end{lem}

\begin{proof}[Path Type Preservation]
  We will use induction on the derivation of $\ev{V;H}{\lv{x}{p}}{\addr{\ga}{r}}$.
  \begin{itemize}
    \item[PE-BASE] Then $p = \base$ and $r = \base$. \\
      By inversion, $V(x)=\ga$. \\
      There is only one rule (\textsc{PT-BASE}) for deriving $\tc{\gG}{\lv{x}{\base}}{\gt}$. \\
      Thus, by inversion, we have $\gG(x)=\gt$. \\
      From $\gG;V\vdash\gS$, we know $\gG(x)=\gS(V(x))$. Thus, $\gt=\gS(\ga)$. \\
      Then by \textsc{RT-BASE}, we have $\tc{\gS;\gY}{\addr{\ga}{\base}}{\gt}$.
    \item[PE-DEOWN] Then $p = \deref{p'}$ and $r = \base$. \\
      By inversion, $\ev{V;H}{\lv{x}{p'}}{\addr{\ga'}{r'}}$
      and $\Read{H}{\addr{\ga'}{r'}}{\own{\ga}}$. \\
      There are two possible rules for deriving $\tc{\gG}{\lv{x}{\deref{p'}}}{\gt}$.
      \begin{itemize}
	\item[PT-DEOWN]
	  By inversion, $\tc{\gG}{\lv{x}{p'}}{\own{\gt}}$. \\
	  By induction, $\tc{\gS;\gY}{\addr{\ga'}{r'}}{\own{\gt}}$. \\
	  By Read Safety, we have $\Read{H}{\addr{\ga'}{r'}}{l}$
	  and $\tc{\gS;\gY}{l}{\own{\gt}}$ for some $l$. \\
	  By Read Uniqueness, we have $l =\ \own{\ga}$. \\
	  There is only one rule (\textsc{LT-OWN}) for deriving 
	  $\tc{\gS;\gY}{\own{\ga}}{\own{\gt}}$. \\
	  Thus, by inversion, $\tc{\gS;\gY}{\addr{\ga}{\base}}{\gt}$.
	\item[PT-DEREF]
	  By inversion, $\tc{\gG}{\lv{x}{p'}}{\tyref{\lt}{q}{\gt}}$. \\
	  By induction, $\tc{\gS;\gY}{\addr{\ga'}{r'}}{\tyref{\lt}{q}{\gt}}$. \\
	  By Read Safety, we have $\Read{H}{\addr{\ga'}{r'}}{l}$
	  and $\tc{\gS;\gY}{l}{\tyref{\lt}{q}{\gt}}$ for some $l$. \\
	  By Read Uniqueness, we have $l =\ \own{\ga}$. \\
	  There are no rules for deriving
	  $\tc{\gS;\gY}{\own{\ga}}{\tyref{\lt}{q}{\gt}}$. \\
	  Thus, by inversion, this case is impossible.
      \end{itemize}
    \item[PE-DEREF] 
      Similar to the \textsc{PE-DEOWN} case, but switching the roles of
      \textsc{PT-DEOWN} and \textsc{PT-DEREF}.
    \item[PE-PROJ] Then $p = \proj{p'}{i}$ and $r = \proj{r'}{i}$. \\
      By inversion, $\ev{V;H}{\lv{x}{p'}}{\addr{\ga}{r'}}$. \\
      There is only one rule (\textsc{PT-PROJ}) for deriving $\tc{\gG}{\lv{x}{\proj{p'}{i}}}{\gt}$. \\
      Thus, by inversion, we have $\gt = \gt_i$ and $\tc{\gG}{\lv{x}{p'}}{\subrec{\gt}{i}{n}}$. \\
      By induction, we have $\tc{\gS;\gY}{\addr{\ga}{r'}}{\subrec{\gt}{i}{n}}$. \\
      Then by \textsc{RT-PROJ}, we have $\tc{\gS;\gY}{\addr{\ga}{\proj{r'}{i}}}{\gt_i}$.
    \item[PE-UNROLL] Then $p = \unroll{\gt'}{p'}$ and $r = \unroll{\gt'}{r'}$. \\
      By inversion, $\ev{V;H}{\lv{x}{p'}}{\addr{\ga}{r'}}$. \\
      There is only one rule (\textsc{PT-UNROLL}) for deriving
      $\tc{\gG}{\lv{x}{\unroll{\gt'}{p'}}}{\gt}$. \\
      Thus, by inversion, we have $\gt' = \tyfix{X}{\gt''}$,
      $\gt = \sub{X}{\tyfix{X}{\gt''}}{\gt''}$, and
      $\tc{\gG}{\lv{x}{p'}}{\tyfix{X}{\gt''}}$. \\
      By induction, we have $\tc{\gS;\gY}{\addr{\ga}{r'}}{\tyfix{X}{\gt''}}$. \\
      Then by \textsc{RT-UNROLL}, we have
      $\tc{\gS;\gY}{\addr{\ga}{\unroll{\tyfix{X}{\gt''}}{r'}}}{\sub{X}{\tyfix{X}{\gt''}}{\gt''}}$.
  \end{itemize}
\end{proof}

\section*{Tracking Initialization}
Types are only half of the information we need to preserve.
To properly type check Patina, we need to know which paths point to initialized memory
and which do not. This is important for ensuring we never use uninitialized memory and
that we correctly free heap memory.
We will accomplish this by creating a shadow heap.
Instead of tracking actual values, we will simply track whether a cell is initialized.
Paths will serve the same function for this shadow heap as routes do for the real heap.

\[
\begin{array}{lccl}
\textrm{Hole} & h & \bnfdef & \uninit \bnfalt \tyint \bnfalt \own{\gs} 
				      \bnfalt \refval{q}{\gt} \bnfalt \subvar{\gt}{i}{n} \\
\textrm{Shadow} & \gs & \bnfdef & h \bnfalt \subrec{\gs}{i}{n} \bnfalt \roll{\gt}{\gs} \\
\end{array}
\]

Since shadows and holes are supposed to be a shadow of the heap,
it is unsurprising that they are very similar to layouts and cells.
However, there are a few key differences.
First, variants are considered an atomic value (and thus a hole) rather than
a structure. This is because the discrimiant and payload of a variant always
share the same initialization state.
Second, while the pointer cells contain routes, the pointer holes do not
contain paths (the analogue of routes).
For references, there is no need because references always point to fully initialized things.
Therefore, the reference hole simply records the type of the reference.
For unique pointers, there is no path that could possibly be used
(otherwise the pointer would not be unique).
Therefore, the unique pointer hole records the shadow of the pointed-to heap memory.

Two examples are useful for understanding shadows.
First, consider nested pointers:
\begin{verbatim}
{
  let x: ~~int = ~~3;
  // Here the shadow of x is `~ ~ int'
  // x, *x, and **x are all initialized

  let y: ~int = *x; // *x is now deinitialized
  // Here the shadow of x is `~ uninit' and the shadow of y is `~ int'
  // x is initialized, but *x and **x are not
  // both y and *y are initialized
}
\end{verbatim}

Second, consider tuples:
\begin{verbatim}
{
  let x: [~int, ~int] = [~1, ~2];
  // Here the shadow of x is [~int, ~int]
  // x, x.1, *(x.1), x.2, and *(x.2) are all initialized

  free(x.1);
  // Here the shadow of x is [uninit, ~int]
  // x, x.1, x.2, and *(x.2) are all initialized, but *(x.1) is not
}
\end{verbatim}

The shadow heap gives the typechecker enough information 
to ensure we do not use or dereference uninitialized memory.
For example, here is how it can catch dereferencing a null pointer:
\begin{verbatim}
{
  let x: ~~int = ~~3;
  free(*x); // deallocates **x, leaving *x uninitialized
  // at this point the shadow of x is `~ uninit'
  let y: int = **x; // ERROR! dereferences a null pointer *x
  // the type checker know the shadow of *x is `uninit'
  // if a further dereference was allowed (i.e. **x), a null pointer derefernce would occur
  // the type checker can avoid null pointer dereferences by forbidding derefs of `uninit' data

}
\end{verbatim}

It can also catch a missing free that would orphan memory.
\begin{verbatim}
{
  let x: ~int = ~3;
  // the shadow of x is `~ int'

  // the stack variable x will be popped off the stack, orphaning the heap memory at *x
  // the type checker can identify this error by seeing that the shadow of x contains `~' data
  // by requiring that shadows cannot contain `~' when freed, the type checker prevents leaks
}
\end{verbatim}

\section*{Checking Shallow Initialization}
We will need an analogue of \textsc{read} for our shadow heap for the
simple purpose of extracting the shadow at a path in the same way we
need to extract the layout at a route from the heap.
However, this analogue also serves as a very useful property checker:
any dereferences in the path are dereferences of initialized memory.
This property, called \emph{shallow initialization} is exactly what we need
for path progress. A shallowly initialized path can successfully dereference
all the pointers necessary to evaluate to a route.
For this reason, we call this operation \textsc{shallow}.
To support dereferencing borrowed references, 
we need to construct a shape from the referenced type.
We do this with the \textsc{init} function.
\newline

\fbox{$\textsc{init} : \textrm{Type} \to \textrm{Shadow}$}

\[
\begin{array}{lcl}
\initShadow{\tyint} &=& \tyint \\
\initShadow{\own{\gt}} &=& \own{\initShadow{\gt}} \\
\initShadow{\tyref{\lt}{q}{\gt}} &=& \refval{q}{\gt} \\
\initShadow{\subrec{\gt}{i}{n}} &=& [\initShadow{\gt_i}]_{i \in \{1\ldots n\}} \\
\initShadow{\subvar{\gt}{i}{n}} &=& \subvar{\gt}{i}{n} \\
\initShadow{\tyfix{X}{\gt}} &=& \roll{\tyfix{X}{\gt}}{\initShadow{\sub{X}{\tyfix{X}{\gt}}{\gt}}} \\
\end{array}
\]

We also need a new context recording the shadows of stack variables.

$$ \gU : \textrm{Variable} \to \textrm{Shadow} $$

\fbox{$\shallow{\gU}{\lv{x}{p}}{\gs}$}

\begin{mathpar}
\infer[SI-BASE]
{\gU(x)=\gs}
{\shallow{\gU}{\lv{x}{\base}}{\gs}}
\and
\infer[SI-DEOWN]
{\shallow{\gU}{\lv{x}{p}}{\own{\gs}}}
{\shallow{\gU}{\lv{x}{\deref{p}}}{\gs}}
\and
\infer[SI-DEREF]
{\shallow{\gU}{\lv{x}{p}}{\refval{q}{\gt}} \\ \initShadow{\gt}=\gs}
{\shallow{\gU}{\lv{x}{\deref{p}}}{\gs}}
\and
\infer[SI-PROJ]
{\shallow{\gU}{\lv{x}{p}}{\subrec{\gs}{i}{n}}}
{\shallow{\gU}{\lv{x}{\proj{p}{i}}}{\gs_i}}
\and
\infer[SI-UNROLL]
{\shallow{\gU}{\lv{x}{p}}{\roll{\gt'}{\gs}}}
{\shallow{\gU}{\lv{x}{\unroll{\gt}{p}}}{\gs}}
\end{mathpar}

\section*{Shadow Typing}
We need to ensure that the shadows assigned to variables are consistent with the types
assigned to variables, i.e. we want to avoid situations like
$\gG(x)=\tyint$ and $\gU(x)=\refval{\qimm}{\tyint}$.
We can easily do this by defining a typing judgement for shadows and
defining a well-formedness condition for $\gU$ that ensures it is consistent with $\gG$.
Similar to typing for layouts, $\uninit$ can only have types that other holes can have.

\fbox{$\tc{}{\gs}{\gt}$}

\begin{mathpar}
\infer[ST-UNINT]
{ }
{\tc{}{\uninit}{\tyint}}
\and
\infer[ST-UNOWN]
{ }
{\tc{}{\uninit}{\own{\gt}}}
\and
\infer[ST-UNREF]
{ }
{\tc{}{\uninit}{\tyref{\lt}{q}{\gt}}}
\and
\infer[ST-UNVAR]
{ }
{\tc{}{\uninit}{\subvar{\gt}{i}{n}}}
\and
\infer[ST-INT]
{ }
{\tc{}{\tyint}{\tyint}}
\and
\infer[ST-OWN]
{\tc{}{\gs}{\gt}}
{\tc{}{\own{\gs}}{\own{\gt}}}
\and
\infer[ST-REF]
{ }
{\tc{}{\refval{q}{\gt}}{\tyref{\lt}{q}{\gt}}}
\and
\infer[ST-VAR]
{ }
{\tc{}{\subvar{\gt}{i}{n}}{\subvar{\gt}{i}{n}}}
\and
\infer[ST-REC]
{\forall i.~\tc{}{\gs_i}{\gt_i}}
{\tc{}{\subrec{\gs}{i}{n}}{\subrec{\gt}{i}{n}}}
\and
\infer[ST-ROLL]
{\tc{}{\gs}{\sub{X}{\tyfix{X}{\gt}}{\gt}}}
{\tc{}{\roll{\tyfix{X}{\gt}}{\gs}}{\tyfix{X}{\gt}}}
\end{mathpar}

\begin{mathpar}
\mprset{flushleft}
\infer
{ 
\dom{\gG}=\dom{\gU} \\\\
\forall x \in \dom{\gG}.~\gG(x)~\textrm{closed} \\\\
\forall x \in \dom{\gG}.~\tc{}{\gU(x)}{\gG(x)}
}
{\tc{}{\gU}{\gG}}
\end{mathpar}

\subsection*{The INIT Function Preserves Types}
Since the \textsc{init} function is supposed to be the
fully initialized shadow of the argument type, it should
be the case that the resulting shadow has that type.
Since \textsc{init} is only defined on closed types, 
we must restrict ourselves to them.
\begin{lem}[\textsc{init} Type Preservation]
  If $\gt$ is closed, then $\tc{}{\initShadow{\gt}}{\gt}$.
\end{lem}

\begin{proof}[\textsc{init} Type Preservation]
  Induction on the form of $\gt$.
  \begin{itemize}
    \item[INT]
      If $\gt = \tyint$, then $\initShadow{\tyint}=\tyint$.
      From \textsc{ST-INT}, we have $\tc{}{\tyint}{\tyint}$.
    \item[OWN]
      If $\gt = \own{\gt'}$, then $\initShadow{\own{\gt'}}=\ \own{\initShadow{\gt'}}$. \\
      Since $\gt$ is closed, $\gt'$ is also closed. \\
      Thus, by induction, $\tc{}{\initShadow{\gt'}}{\gt'}$. \\
      From \textsc{ST-OWN}, we have $\tc{}{\own{\initShadow{\gt'}}}{\own{\gt'}}$.
    \item[REF]
      If $\gt = \tyref{\lt}{q}{\gt'}$, 
      then $\initShadow{\tyref{\lt}{q}{\gt'}}=\refval{q}{\gt'}$. \\
      $\tc{}{\refval{q}{\gt'}}{\tyref{\lt}{q}{\gt'}}$ follows immediately by \textsc{ST-REF}.
    \item[REC]
      If $\gt = \subrec{\gt'}{i}{n}$,
      then $\initShadow{\subrec{\gt'}{i}{n}}=[\initShadow{\gt'_i}]_{i\in\{1\ldots n\}}$. \\
      Since $\gt$ is closed, $\gt'_i$ is closed for all $i$. \\
      Thus, by induction, for all $i$ we have $\tc{}{\initShadow{\gt'_i}}{\gt'_i}$. \\
      Then by \textsc{ST-REC}, we have 
      $\tc{}{[\initShadow{\gt'_i}]_{i\in\{1\ldots n\}}}{\subrec{\gt'}{i}{n}}$.
    \item[VAR]
      If $\gt = \subvar{\gt'}{i}{n}$,
      then $\initShadow{\subvar{\gt'}{i}{n}}=\subvar{\gt'}{i}{n}$. \\
      $\tc{}{\subvar{\gt'}{i}{n}}{\subvar{\gt'}{i}{n}}$ follows immediately from \textsc{ST-VAR}.
    \item[FIX]
      If $\gt = \tyfix{X}{\gt'}$,
      then $\initShadow{\tyfix{X}{\gt'}} 
         = \roll{\tyfix{X}{\gt'}}{\initShadow{\sub{X}{\tyfix{X}{\gt'}}{\gt'}}}$. \\
      Since $\gt$ is closed, $\gt'$ might not be closed;
      however, the only possible free variable is $X$. \\
      Thus, the unrolling $\sub{X}{\tyfix{X}{\gt'}}{\gt'}$ is closed. \\
      Hence, by induction, 
      $\tc{}{\initShadow{\sub{X}{\tyfix{X}{\gt'}}{\gt'}}}{\sub{X}{\tyfix{X}{\gt'}}{\gt'}}$. \\
      Then by \textsc{ST-ROLL}, we have
      $\tc{}{\roll{\tyfix{X}{\gt'}}{\initShadow{\sub{X}{\tyfix{X}{\gt'}}{\gt'}}}}
	    {\tyfix{X}{\gt'}}$.
  \end{itemize}
\end{proof}

\subsection*{Shallow Preserves Types}
Just as with \textsc{read}, \textsc{shallow} preserves types.
This lemma is used primarily to demonstrate the impossibility 
of certain subcases of dereferences in later proofs.

\begin{lem}[\textsc{shallow} Type Preservation]
  If $\tc{}{\gU}{\gG}$, $\tc{\gG}{\lv{x}{p}}{\gt}$, and $\shallow{\gU}{\lv{x}{p}}{\gs}$
  then $\tc{}{\gs}{\gt}$.
\end{lem}

\begin{proof}[\textsc{shallow} Type Preservation]
  Induction on the derivation of $\shallow{\gU}{\lv{x}{p}}{\gs}$.
  \begin{itemize}
    \item[SI-BASE] Then $p = \base$. \\
      By inversion, $\gU(x)=\gs$. \\
      There is only one rule (\textsc{PT-BASE}) for deriving $\tc{\gG}{\lv{x}{\base}}{\gt}$. \\
      Thus, by inversion, we have $\gG(x)=\gt$. \\
      From $\tc{}{\gU}{\gG}$, we have $\tc{}{\gU(x)}{\gG(x)}$. \\
      Ergo, we have $\tc{}{\gs}{\gt}$.
    \item[SI-DEOWN] Then $p = \deref{p'}$. \\
      By inversion, $\shallow{\gU}{\lv{x}{p'}}{\own{\gs}}$. \\
      There are two possible rules for deriving $\tc{\gG}{\lv{x}{\deref{p'}}}{\gt}$.
      \begin{itemize}
	\item[PT-DEOWN]
	  By inversion, $\tc{\gG}{\lv{x}{p'}}{\own{\gt}}$. \\
	  By induction, $\tc{}{\own{\gs}}{\own{\gt}}$. \\
	  There is only one rule (\textsc{ST-OWN}) for deriving $\tc{}{\own{\gs}}{\own{\gt}}$. \\
	  Thus, by inversion, $\tc{}{\gs}{\gt}$.
	\item[PT-DEREF]
	  By inversion, $\tc{\gG}{\lv{x}{p'}}{\tyref{\lt}{q}{\gt}}$. \\
	  By induction, $\tc{}{\own{\gs}}{\tyref{\lt}{q}{\gt}}$. \\
	  There are no rules for deriving $\tc{}{\own{\gs}}{\tyref{\lt}{q}{\gt}}$. \\
	  Thus, by inversion, this case is impossible.
      \end{itemize}
    \item[SI-DEREF] Then $p = \deref{p'}$. \\
      By inversion, $\shallow{\gU}{\lv{x}{p'}}{\refval{q}{\gt}}$ and $\initShadow{\gt}=\gs$. \\
      There are two possible rules for deriving $\tc{\gG}{\lv{x}{\deref{p'}}}{\gt}$.
      \begin{itemize}
	\item[PT-DEOWN]
	  By inversion, $\tc{\gG}{\lv{x}{p'}}{\own{\gt}}$. \\
	  By induction, $\tc{}{\refval{q}{\gt}}{\own{\gt}}$. \\
	  There are no rules for deriving $\tc{}{\refval{q}{\gt}}{\own{\gt}}$. \\
	  Thus, by inversion, this case is impossible.
	\item[PT-DEREF]
	  By inversion, $\tc{\gG}{\lv{x}{p'}}{\tyref{\lt}{q}{\gt}}$. \\
	  Since $\tc{}{\gU}{\gG}$ guarantees that $\gG(x)$ is closed
	  and a simple inspection of the path typing rules shows that
	  this implies all well typed paths have closed types,
	  we know that $\gt$ is closed. \\
	  By \textsc{init} Type Preservation, $\textsc{init}(\gt)=\gs$ implies $\tc{}{\gs}{\gt}$.
      \end{itemize}
    \item[SI-PROJ] Then $p = \proj{p'}{i}$ and $\gs = \gs_i$. \\
      By inversion, $\shallow{\gU}{\lv{x}{p'}}{\subrec{\gs}{i}{n}}$. \\
      There is only one rule (\textsc{PT-PROJ}) for deriving 
      $\tc{\gG}{\lv{x}{\proj{p'}{i}}}{\gt}$. \\
      Thus, by inversion, we have $\gt=\gt_i$ and $\tc{\gG}{\lv{x}{p'}}{\subrec{\gt}{i}{n}}$. \\
      By induction, we have $\tc{}{\subrec{\gs}{i}{n}}{\subrec{\gt}{i}{n}}$. \\
      There is only one rule (\textsc{ST-REC}) for 
      deriving $\tc{}{\subrec{\gs}{i}{n}}{\subrec{\gt}{i}{n}}$. \\
      Thus, by inversion, we have $\tc{}{\gs_i}{\gt_i}$.
    \item[SI-UNROLL] Then $p = \unroll{\gt'}{p'}$. \\
      By inversion, $\shallow{\gU}{\lv{x}{p'}}{\roll{\gt''}{\gs}}$. \\
      There is only one rule (\textsc{PT-UNROLL}) for deriving
      $\tc{\gG}{\lv{x}{\unroll{\gt'}{p'}}}{\gt}$. \\
      By inversion, $\gt' = \tyfix{X}{\gt'''}$, $\gt = \sub{X}{\tyfix{X}{\gt'''}}{\gt'''}$,
      and $\tc{\gG}{\lv{x}{p'}}{\tyfix{X}{\gt'''}}$. \\
      By induction, $\tc{}{\roll{\gt''}{\gs}}{\tyfix{X}{\gt'''}}$. \\
      There is only one rule (\textsc{ST-ROLL}) for deriving this. \\
      Thus, by inversion, we have $\gt'' = \tyfix{X}{\gt'''}$ and 
      $\tc{}{\gs}{\sub{X}{\tyfix{X}{\gt'''}}{\gt'''}}$.
  \end{itemize}
\end{proof}

\section*{Shadow Coherence}
Of course, we also need to check that our shadow heap correctly models
the actual heap. For the most part, it is exactly the relation you would expect.
The use of \textsc{init} in \textsc{LS-REF} and \textsc{LS-VAR} ensures that
the pointees of borrowed references and the payload of initialized variants
are fully initialized.
The $\textsc{droppable}(l)$ judgment checks that $l$ owns no initialized heap memory,
i.e. $\own{\addr{\ga}{r}}$ is not reachable from $l$ except through references.
These layouts are safe to deallocate because we cannot orphan heap memory by doing so.
The $\tc{V}{H}{\gU}$ judgment ensures that the shadow heap corresponds to the real heap.
\newline

\fbox{$\textsc{droppable}(l)$}

\begin{mathpar}
\infer[D-VOID]
{ }
{\textsc{droppable}(\void)}
\and
\infer[D-INT]
{ }
{\textsc{droppable}(z)}
\and
\infer[D-REF]
{ }
{\textsc{droppable}(\refval{q}{\addr{\ga}{r}})}
\and
\infer[D-VAR]
{\textsc{droppable}(c) \\ \textsc{droppable}(l)}
{\textsc{droppable}(\varval{c}{l})}
\and
\infer[D-REC]
{\forall i.~\textsc{droppable}(l_i)}
{\textsc{droppable}(\subrec{l}{i}{n})}
\and
\infer[D-ROLL]
{\textsc{droppable}(l)}
{\textsc{droppable}(\roll{\gt}{l})}
\end{mathpar}

\fbox{$\tc{H}{l}{\gs}$}

\begin{mathpar}
\infer[LS-VOID]
{ }
{\tc{H}{\void}{\uninit}}
\and
\infer[LS-INT]
{ }
{\tc{H}{z}{\tyint}}
\and
\infer[LS-OWN]
{\Read{H}{\addr{\ga}{\base}}{l} \\ \tc{H}{l}{\gs}}
{\tc{H}{\own{\ga}}{\own{\gs}}}
\and
\infer[LS-REF]
{\Read{H}{\addr{\ga}{r}}{l} \\ \tc{H}{l}{\initShadow{\gt}}}
{\tc{H}{\refval{q}{\addr{\ga}{r}}}{\refval{q}{\gt}}}
\and
\infer[LS-VOIDVAR]
{ }
{\tc{H}{\varval{\void}{\void}}{\uninit}}
\and
\infer[LS-DROPVAR]
{\textsc{droppable}(l)}
{\tc{H}{\varval{i}{l}}{\uninit}}
\and
\infer[LS-VAR]
{\tc{H}{l}{\initShadow{\gt_i}}}
{\tc{H}{\varval{i}{l}}{\subvar{\gt}{i}{n}}}
\and
\infer[LS-REC]
{\forall i.~\tc{H}{l_i}{\gs_i}}
{\tc{H}{\subrec{l}{i}{n}}{\subrec{\gs}{i}{n}}}
\and
\infer[LS-ROLL]
{\tc{H}{l}{\gs}}
{\tc{H}{\roll{\tyfix{X}{\gt}}{l}}{\roll{\tyfix{X}{\gt}}{\gs}}}
\end{mathpar}

\begin{mathpar}
\mprset{flushleft}
\infer
{ 
\dom{V}=\dom{\gU} \\\\
\dom H = \textsc{reachable}(V,H) \\\\
\forall x \in \dom{V}.~\tc{H}{H(V(x))}{\gU(x)}
}
{\tc{V}{H}{\gU}}
\end{mathpar}

\section*{Shadow Preservation}
The second main preservation lemma for our core operations.
The initialization data, in the form of a shadow,
is preserved by path evaulation and \textsc{read}
so that the resulting layout is correctly described by the shadow.

\begin{lem}[Shadow Preservation]
  If $\tc{V}{H}{\gU}$, $\shallow{\gU}{\lv{x}{p}}{\gs}$, 
  $\ev{V;H}{\lv{x}{p}}{\addr{\ga}{r}}$ , and $\Read{H}{\addr{\ga}{r}}{l}$
  then $\tc{H}{l}{\gs}$.
\end{lem}

\begin{proof}[Shadow Preservation]
  Induction on the derivation fo $\ev{V;H}{\lv{x}{p}}{\addr{\ga}{r}}$.
  \begin{itemize}
    \item[\textsc{PE-BASE}] 
      Then $p = \base$ and $r = \base$.\\
      By inversion, $V(x)=\ga$.\\
      There is only one rule (\textsc{SI-BASE}) for deriving $\shallow{\gU}{\lv{x}{\base}}{\gs}$.\\
      Thus, by inversion, $\gU(x)=\gs$.\\
      There is only one rule (\textsc{RD-BASE}) for deriving $\Read{H}{\addr{\ga}{\base}}{l}$.\\
      Thus, by inversion, $H(\ga)=l$.\\
      By $\tc{V}{H}{\gU}$, we have $\tc{H}{H(V(x))}{\gU(x)}$. \\
      However, this is just $\tc{H}{l}{\gs}$ as required.
    \item[\textsc{PE-DEOWN}] 
      Then $p = \deref{p'}$ and $r = \base$.\\
      By inversion, $\ev{V;H}{\lv{x}{p'}}{\addr{\ga'}{r'}}$ 
      and $\Read{H}{\addr{\ga'}{r'}}{\own{\ga}}$. \\
      There are two possible rules for deriving $\shallow{\gU}{\lv{x}{\deref{p'}}}{\gs}$.
      \begin{itemize}
	\item[\textsc{SI-DEOWN}]
	  By inversion, $\shallow{\gU}{\lv{x}{p'}}{\own{\gs}}$.\\
	  By induction, $\tc{H}{\own{\ga}}{\own{\gs}}$.\\
	  There is only one rule (\textsc{LS-OWN}) for deriving this.\\
	  Thus, by inversion, we have $\Read{H}{\addr{\ga}{\base}}{l'}$ and $\tc{H}{l'}{\gs}$.\\
	  By Read Uniqueness, we have $l' = l$.\\
	  Ergo, $\tc{H}{l}{\gs}$ as required.
	\item[\textsc{SI-DEREF}]
	  By inversion, $\shallow{\gU}{\lv{x}{p'}}{\refval{q}{\gt}}$ 
	  and $\initShadow{\gt}=\gs$.\\
	  By induction, $\tc{H}{\own{\ga}}{\refval{q}{\gt}}$.\\
	  There are no rules for deriving this.\\
	  Thus, by inversion, this case is impossible.
      \end{itemize}
    \item[\textsc{PE-DEREF}] 
      Then $p = \deref{p'}$.\\
      By inversion, $\ev{V;H}{\lv{x}{p'}}{\addr{\ga'}{r'}}$
      and $\Read{H}{\addr{\ga'}{r'}}{\refval{q}{\addr{\ga}{r}}}$.\\
      There are two possible rules for deriving $\shallow{\gU}{\lv{x}{\deref{p'}}}{\gs}$.
      \begin{itemize}
	\item[\textsc{SI-DEOWN}]
	  By inversion, $\shallow{\gU}{\lv{x}{p'}}{\own{\gs}}$.\\
	  By induction, $\tc{H}{\refval{q}{\addr{\ga}{r}}}{\own{\gs}}$.\\
	  There are no rules for deriving this.\\
	  Thus, by inversion, this case is impossible.
	\item[\textsc{SI-DEREF}]
	  By inversion, $\shallow{\gU}{\lv{x}{p'}}{\refval{q}{\gt}}$ 
	  and $\initShadow{\gt}=\gs$.\\
	  By induction, $\tc{H}{\refval{q}{\addr{\ga}{r}}}{\refval{q}{\gt}}$.\\
	  There is only one rule (\textsc{LS-REF}) for deriving this.\\
	  Thus, by induction, we have $\Read{H}{\addr{\ga}{r}}{l'}$
	  and $\tc{H}{l'}{\initShadow{\gt}}$.\\
	  By Read Uniqueness, we have $l' = l$.\\
	  Ergo, $\tc{H}{l}{\gs}$ as required.
      \end{itemize}
    \item[\textsc{PE-PROJ}] 
      Then $p = \proj{p'}{i}$ and $r = \proj{r'}{i}$.\\
      By inversion, $\ev{V;H}{\lv{x}{p'}}{\addr{\ga}{r'}}$.\\
      There is only one rule (\textsc{SI-PROJ}) for 
      deriving $\shallow{\gU}{\lv{x}{\proj{p'}{i}}}{\gs}$.\\
      Thus, by inversion, we have $\gs=\gs_i$ and
      $\shallow{\gU}{\lv{x}{p'}}{\subrec{\gs}{i}{n}}$.\\
      There is only one rule (\textsc{RD-PROJ}) for 
      deriving $\Read{H}{\addr{\ga}{\proj{r'}{i}}}{l}$.\\
      Thus, by inversion, we have $l=l_i$ and
      $\Read{H}{\addr{\ga}{r'}}{\subrec{l}{i}{n}}$.\\
      By induction, we have $\tc{H}{\subrec{l}{i}{n}}{\subrec{\gs}{i}{n}}$.\\
      There is only one rule (\textsc{LS-REC}) for deriving this.\\
      Thus, by inversion, we have $\tc{H}{l_i}{\gs_i}$ as required.
    \item[\textsc{PE-UNROLL}] 
      Then $p = \unroll{\gt}{p'}$ and $r = \unroll{\gt}{r'}$.\\
      By inversion, $\ev{V;H}{\lv{x}{p'}}{\addr{\ga}{r'}}$.\\
      There is only one rule (\textsc{SI-UNROLL}) for
      deriving $\shallow{\gU}{\lv{x}{\unroll{\gt}{p'}}}{\gs}$.\\
      Thus, by inversion, we have $\shallow{\gU}{\lv{x}{p'}}{\roll{\gt'}{\gs}}$.\\
      There is only one rule (\textsc{RD-UNROLL}) for
      deriving $\Read{H}{\addr{\ga}{\unroll{\gt}{r'}}}{l}$.\\
      Thus, by inversion, we have $\Read{H}{\addr{\ga}{r'}}{\roll{\gt''}{l}}$.\\
      By induction, $\tc{H}{\roll{\gt''}{l}}{\roll{\gt'}{\gs}}$.\\
      There is only one rule (\textsc{LS-ROLL}) for deriving this.\\
      Thus, by inversion, we have $\gt' = \gt'' = \tyfix{X}{\gt'''}$
      and $\tc{H}{l}{\gs}$ required.
  \end{itemize}
\end{proof}

\section*{Path Progress}
Finally, we have all we need to proof path evaluation progress under the right circumstances.
Assuming all our contexts and the heap are well-formed,
then a shallowly initialized, well-typed path can evaluate to some route.
We can then rely to path preservation to show that route has the same type as the path.
Read safety then lets us show that we can read a layout at that route
and that it has the same type. Shadow preservation guarantees us that our
model of the layout reflects reality. All of the key pieces we need for
the higher layers of Patina.

\begin{lem}[Path Progress]
  If $\tc{}{H}{\gS;\gY}$, $\gG;V\vdash\gS$, $\tc{}{\gU}{\gG}$, $\tc{V}{H}{\gU}$,
  $\tc{\gG}{\lv{x}{p}}{\gt}$, and $\shallow{\gU}{\lv{x}{p}}{\gs}$
  then there is some $\addr{\ga}{r}$ such that $\ev{V;H}{\lv{x}{p}}{\addr{\ga}{r}}$.
\end{lem}

\begin{proof}[Path Progress]
  Induction on the derivation of $\tc{\gG}{\lv{x}{p}}{\gt}$.
  \begin{itemize}
    \item[PT-BASE] 
      Then $p = \base$.\\
      By inversion, $\gG(x)=\gt$.\\
      From $\gG;V\vdash\gS$, we have $\gG(x)=\gS(V(x))$.\\
      Ergo, $\exists \ga.~V(x)=\ga$.\\
      Then by \textsc{PE-BASE}, we have $\ev{V;H}{\lv{x}{\base}}{\addr{\ga}{\base}}$.
    \item[PT-DEOWN] 
      Then $p = \deref{p'}$.\\
      By inversion, $\tc{\gG}{\lv{x}{p'}}{\own{\gt}}$.\\
      There are two possible rules for deriving $\shallow{\gU}{\lv{x}{\deref{p'}}}{\gs}$.
      \begin{itemize}
	\item[\textsc{SI-DEOWN}]
	  By inversion, $\shallow{\gU}{\lv{x}{p'}}{\own{\gs}}$.\\
	  By induction, $\exists \addr{\ga}{r}.~\ev{V;H}{\lv{x}{p'}}{\addr{\ga}{r}}$.\\
	  By Path Type Preservation, $\tc{\gS;\gY}{\addr{\ga}{r}}{\own{\gt}}$.\\
	  By Read Safety, $\Read{H}{\addr{\ga}{r}}{l}$ 
	  and $\tc{\gS;\gY}{l}{\own{\gt}}$ for some $l$.\\
	  By Shadow Preservation, $\tc{H}{l}{\own{\gs}}$.\\
	  There is only one rule (\textsc{LS-OWN}) for deriving this.\\
	  Thus, by inversion, we have $l =\ \own{\ga'}$,
	  $\Read{H}{\addr{\ga'}{\base}}{l'}$, and $\tc{H}{l'}{\gs}$.\\
	  Using \textsc{PE-DEOWN} with
	  $\ev{V;H}{\lv{x}{p'}}{\addr{\ga}{r}}$ and
	  $\Read{H}{\addr{\ga}{r}}{\own{\ga'}}$, \\ 
	  we get $\ev{V;H}{\lv{x}{\deref{p'}}}{\addr{\ga'}{\base}}$,
	  which is what was required.
	\item[\textsc{SI-DEREF}]
	  By inversion, $\shallow{\gU}{\lv{x}{p'}}{\refval{q}{\gt}}$.\\
	  By \textsc{shallow} Type Preservation, $\tc{}{\refval{q}{\gt}}{\own{\gt}}$.\\
	  There are no rules for deriving this.\\
	  Thus, by inversion, this case is impossible.
      \end{itemize}

    \item[PT-DEREF] 
      Then $p = \deref{p'}$.\\
      By inversion, $\tc{\gG}{\lv{x}{p'}}{\tyref{\lt}{q}{\gt}}$.\\
      There are two possible rules for deriving $\shallow{\gU}{\lv{x}{\deref{p'}}}{\gs}$.
      \begin{itemize}
	\item[\textsc{SI-DEOWN}]
	  By inversion, $\shallow{\gU}{\lv{x}{p'}}{\own{\gs}}$.\\
	  By \textsc{shallow} Type Preservation, $\tc{}{\own{\gs}}{\tyref{\lt}{q}{\gt}}$.\\
	  There are no rules for deriving this. \\
	  Thus, by inversion, this case is impossible.
	\item[\textsc{SI-DEREF}]
	  By inversion, $\shallow{\gU}{\lv{x}{p'}}{\refval{q}{\gt}}$.\\
	  By induction, $\exists \addr{\ga}{r}.~\ev{V;H}{\lv{x}{p'}}{\addr{\ga}{r}}$.\\
	  By Path Type Preservation, $\tc{\gS;\gY}{\addr{\ga}{r}}{\tyref{\lt}{q}{\gt}}$.\\
	  By Read Safety, $\Read{H}{\addr{\ga}{r}}{l}$ 
	  and $\tc{\gS;\gY}{l}{\tyref{\lt}{q}{\gt}}$ for some $l$.\\
	  By Shadow Preservation, $\tc{H}{l}{\refval{q}{\gt}}$.\\
	  There is only one rule (\textsc{LS-REF}) for deriving this.\\
	  Thus, by inversion, we have $l = \refval{q}{\addr{\ga'}{r'}}$
	  and $\Read{H}{\addr{\ga'}{r'}}{l'}$.\\
	  Using \textsc{PE-DEREF} with 
	  $\ev{V;H}{\lv{x}{p'}}{\addr{\ga}{r}}$ and
	  $\Read{H}{\addr{\ga}{r}}{\refval{q}{\addr{\ga'}{r'}}}$,\\
	  we get $\ev{V;H}{\lv{x}{\deref{p'}}}{\addr{\ga'}{r'}}$,
	  which is what was required.
      \end{itemize}

    \item[PT-PROJ] 
      Then $p = \proj{p'}{i}$ and $\gt=\gt_i$.\\
      By inversion, $\tc{\gG}{\lv{x}{p'}}{\subrec{\gt}{i}{n}}$.\\
      There is only one rule (\textsc{SI-PROJ}) for 
      deriving $\shallow{\gU}{\lv{x}{\deref{p'}}}{\gs}$.\\
      Thus, by inversion, we have $\gs=\gs_i$ 
      and $\shallow{\gU}{\lv{x}{p'}}{\subrec{\gs}{i}{n}}$.\\
      By induction, $\exists \addr{\ga}{r}.~\ev{V;H}{\lv{x}{p'}}{\addr{\ga}{r}}$.\\
      Then by \textsc{PE-PROJ}, we have $\ev{V;H}{\lv{x}{\proj{p'}{i}}}{\addr{\ga}{\proj{r}{i}}}$.
    \item[PT-UNROLL] 
      Then $p = \unroll{\tyfix{X}{\gt'}}{p'}$ and $\gt = \sub{X}{\tyfix{X}{\gt'}}{\gt'}$. \\
      By inversion, $\tc{\gG}{\lv{x}{p'}}{\tyfix{X}{\gt'}}$.\\
      There is only one rule (\textsc{SI-UNROLL}) for 
      deriving $\shallow{\gU}{\lv{x}{\unroll{\tyfix{X}{\gt'}}{p'}}}{\gs}$.\\
      Thus, by inversion, $\shallow{\gU}{\lv{x}{p'}}{\roll{\gt''}{\gs}}$.\\
      By induction, $\exists \addr{\ga}{r}.~\ev{V;H}{\lv{x}{p'}}{\addr{\ga}{r}}$.\\
      Then by \textsc{PE-UNROLL}, we have 
      $\ev{V;H}{\lv{x}{\unroll{\tyfix{X}{\gt'}}{p'}}}{\addr{\ga}{\unroll{\tyfix{X}{\gt'}}{r'}}}$.
  \end{itemize}
\end{proof}

\section*{R-Values}

R-values in Patina are expressions that can evaluate to a layout.
Effectively, they are all the ways to construct values that can be deconstructed by L-values.
This layer of Patina involves the borrow checker, which ensures the borrowed reference guarantees.
In order to focus on the details of the borrow checker, we will pare down our types.
Variants, tuples, and recursive types have straightforward behavior here 
and would unnecessarily clutter the discussion.
Going forward, we will use this subset of our earlier language:

\[
\begin{array}{lccl}
\textrm{Context}  & \gG & : & \textrm{Variable} \rightarrow \textrm{Type} \\
\textrm{Shadow Heap} & \gU & : & \textrm{Variable} \rightarrow \textrm{Shadow} \\
\textrm{Map} & V   & : & \textrm{Variable} \rightarrow \textrm{Allocation} \\
\textrm{Heap Type} & \gS & : & \textrm{Allocation} \rightarrow \textrm{Type} \\
\textrm{Heap} & H   & : & \textrm{Allocation} \rightarrow \textrm{Layout} \\
& & & \\
\textrm{Lifetime} & \lt & & \\
\textrm{Variable} & x & & \\
\textrm{Integer} & z & & \\
\textrm{Allocation} & \ga & & \\
& & & \\
\textrm{Qualifier} & q & \bnfdef & \qimm \bnfalt \qmut \\
\textrm{Type} & \gt & \bnfdef & \tyint \bnfalt \own{\gt} \bnfalt \tyref{\lt}{q}{\gt} \\
\textrm{Path} & p & \bnfdef & \base \bnfalt \deref{p} \\
\textrm{Cell} & c & \bnfdef & \void \bnfalt z \bnfalt \own{\ga} \bnfalt \refval{q}{\ga} \\
\textrm{Layout} & l & \bnfdef & c \\
\textrm{Hole} & h & \bnfdef & \uninit \bnfalt \tyint \bnfalt \own{\gs} \bnfalt \refval{q}{\gt} \\
\textrm{Shadow} & \gs & \bnfdef & h \\
\end{array}
\]

Routes would only consist of $\base$, so they are now elided.
Without variants, the $\gY$ context is irrelevant, so it is elided as well.

We define three kinds of expressions: integer constants, using the value at a L-value, 
and creating a borrowed reference to the value at a L-value.

\[
\begin{array}{lccl}
\textrm{Expression} & e & \bnfdef & z \bnfalt \lv{x}{p} \bnfalt \tyref{\lt}{q}{\lv{x}{p}}
\end{array}
\]

\subsection*{Tracking Loans}

When we use a L-value in an expression, we must ensure the operation we are performing on it
does not violate the promises of existing loans, and we must ensure that the value it refers
to is fully initialized. When we create a borrowed reference, we must ensure that the
promises of the borrow can actually be upheld for the duration of the borrow.
In order to make these checks, we need to know what is borrowed, how it is borrowed, and for
how long it is borrowed. We will accomplish this by adding a stack of loans to every
level of the shadow heap:

\[
\begin{array}{lccl}
\textrm{Hole} & h & \bnfdef & \uninit \bnfalt \tyint \bnfalt \own{\gs} \bnfalt \refval{q}{\gs} \\
\textrm{Bank} & \$ & \bnfdef & \emptyset \bnfalt \$, (\lt,q) \\
\textrm{Shadow} & \gs & \bnfdef & \$:h \\
\end{array}
\]

Each pair $(\lt, q)$ describes the duration and manner of a loan, 
with more recent loans at the front of the stack.
Since every path can be borrowed, every path must have a loan stack (or bank) associated with it.
The easiest way to do that is to tag every shadow with a bank.
Note that the borrowed reference hole has changed.
When we only cared about initialization, we used the hole $\refval{q}{\gt}$.
Since everything reachable via the borrow was initialized, we did not need to record
the initialization state.
Now, however, we do need to record the loan state of things reachable via the borrow.
Thus, we make the reference hole recursive: $\refval{q}{\gs}$.

\subsection*{Shadow Typing with Banks}
Since our shadows now include loan information, we need to modify our shadow typing.
While we have include lifetimes in our model from the beginning, this is the first time
we will actually utilize them. Before we can discuss how to check banks, we must first
discuss what lifetimes are, where they come from, and what they give us.

\subsubsection*{The Lifetime Relation}
Lifetimes, as the name implies, encompass a notion of duration.
Specifically, they describe the duration a value is allocated and 
the duration of a loan. Each new block of scope defines a new lifetime.
So there is a clear ordering on lifetimes, $\lt \leq \lt'$ if the duration of $\lt$
is not longer than the duration of $\lt'$. More formally:

$$ \ltr \subseteq \textrm{Lifetime} \times \textrm{Lifetime} $$

$$ \infer{\ltr \ \textrm{is a poset}}{\vdash \ltr} $$

where $(\lt, \lt') \in \ltr$ means $\lt \leq \lt'$. Rust allows some other sources of lifetimes
not relevant to the current model: lifetime parameters for functions and \emph{static}, which
is the oldest lifetime and typically used for string constants. Though the lifetimes in our
model form a total order, the general Rust lifetime relation is only a partial order.
Since we intend this model to be a potential basis for all of Rust, we will ignore the
totality of our lifetimes.

\subsubsection*{Bank Coherence}
As mentioned before, lifetimes come from scope blocks.
This means that every variable has a lifetime, which comes from the block it was defined in.
Since variables are deallocated at the end of the block in which they are defined,
that lifetime is a accurate model. We record the lifetimes of variables in a new context:

$$ L : \textrm{Variable} \rightarrow \textrm{Lifetime} $$

Furthermore, if variables only exist for a particular lifetime, then it makes sense that
loans of memory owned by the variable cannot be longer than the lifetime of the variable.
Otherwise the deallocation of the variable could create dangling pointers elsewhere.

Finally, we must ensure that more recent loans are not inconsistent with older loans
of the same path. If a path is already loaned out for some lifetime $\lt$ and
we loan it out again for some lifetime $\lt'$, then $(\lt', \lt) \in \ltr$.
A loan only controls a value during the loan's lifetime, so it cannot make promises
for lifetimes longer than that. Additionally, given a mutable loan we can reloan it immutably,
but not the other way. A mutable loan knows it has unique access to a value, so it can
guarantee the value can be immutable if it wishes. On the other hand, an immutable loan
promises that the value will not be changed for its duration, which violates the requirement
that a mutable loan should be able to mutate its referent. We characterize this relationship
with an ordering on qualifiers: $\qimm < \qmut$.
\newline

\fbox{$\bankwf{\ltr}{\$}{\lt}$}

\begin{mathpar}
\infer[BWF-NONE]{ }{\bankwf{\ltr}{\emptyset}{\lt}} \and
\infer[BWF-ONE]{(\lt', \lt) \in \ltr}{\bankwf{\ltr}{\emptyset , (\lt', q)}{\lt}} \and
\infer[BWF-TWO]
  {(\lt_2, \lt_1) \in \ltr \\ q_2 \leq q_1 \\ \bankwf{\ltr}{\$, (\lt_1, q_1)}{\lt}}
  {\bankwf{\ltr}{\$,(\lt_1, q_1), (\lt_2, q_2)}{\lt}} \and
\end{mathpar}

\subsubsection*{Bounded Shadow Typing}
We need to update our shadow typing judgment to support checking bank coherence,
but we need to check two new things as well. First, we will require that
variables can only reference memory that they will not outlast. This means
we require borrowed references to have lifetimes not shorter than the bounding lifetime,
which is the reverse of the requirement on loans. This should make sense, we can only use
memory that will exist at least as long as us, but we can only loan out our memory
to others that will not last longer than us.

Second, we need to make sure that more specific loans do not contradict more general loans.
For example, if a variable $x$ is loaned immutably, then a mutable loan of $\deref{x}$
would not make sense. The immutable loan of $x$ promises that no memory reachable via $x$
will change, but the mutable loan of $\deref{x}$ promises that the memory reachable 
via $\deref{x}$ can be changed through the borrowed reference. Similarly, if $x$ was loaned
mutably, then $\deref{x}$ could not be loaned at all. The loan of $x$ promises that only
the borrowed reference has access to memory reachable from $x$, but the loan of $\deref{x}$
creates an alias to part of that memory. In general, a wider immutable loan requires more
specific loans to not be mutable loans, and a wider mutable loan prevents any more specific
loans.

\[
\begin{array}{lccl}
\textrm{Maybe Qualifier} & mq & \bnfdef & \mqnone \bnfalt q
\end{array}
\]

\fbox{$\nm{\$}$}

\begin{mathpar}
  \infer[NM-EMPTY]{ }{\nm{\emptyset}} \and
  \infer[NM-IMM]{ }{\nm{\$, (\lt, \qimm)}}
\end{mathpar}

\fbox{$\tcb{\ltr}{\gs}{\gt}{mq}{\lt}$}

\begin{mathpar}
  \infer[ST-NONE]
    {\bankwf{\ltr}{\emptyset}{\lt} \\ \tcb{\ltr}{h}{\gt}{\mqnone}{\lt}}
    {\tcb{\ltr}{(\emptyset : h)}{\gt}{\mqnone}{\lt}} \and
  \infer[ST-SOME]
    {\bankwf{\ltr}{\$,(\lt',q)}{\lt} \\ \tcb{\ltr}{h}{\gt}{q}{\lt}}
    {\tcb{\ltr}{(\$,(\lt',q) : h)}{\gt}{\mqnone}{\lt}} \and
  \infer[ST-IMM]
    {\bankwf{\ltr}{\$}{\lt} \\ \nm{\$} \\ \tcb{\ltr}{h}{\gt}{\qimm}{\lt}}
    {\tcb{\ltr}{(\$ : h)}{\gt}{\qimm}{\lt}} \and
  \infer[ST-MUT]
    {\bankwf{\ltr}{\emptyset}{\lt} \\ \tcb{\ltr}{h}{\gt}{\qmut}{\lt}}
    {\tcb{\ltr}{(\emptyset : h)}{\gt}{\qmut}{\lt}}
\end{mathpar}

\fbox{$\tcb{\ltr}{h}{\gt}{mq}{\lt}$}

\begin{mathpar}
  \infer[HT-UNINT]{ }{\tcb{\ltr}{\uninit}{\tyint}{mq}{\lt}} \and
  \infer[HT-UNOWN]{ }{\tcb{\ltr}{\uninit}{\own{\gt}}{mq}{\lt}} \and
  \infer[HT-UNREF]{(\lt,\lt')\in\ltr}{\tcb{\ltr}{\uninit}{\tyref{\lt'}{q}{\gt}}{mq}{\lt}} \and
  \infer[HT-INT]{ }{\tcb{\ltr}{\tyint}{\tyint}{mq}{\lt}} \and
  \infer[HT-OWN]{\tcb{\ltr}{\gs}{\gt}{mq}{\lt}}{\tcb{\ltr}{\own{\gs}}{\own{\gt}}{mq}{\lt}} \and
  %\infer[HT-REF]
    %{(\lt, \lt') \in \ltr \\ \tcb{\ltr}{\gs}{\gt}{mq}{\lt} \\ \gs\ \textrm{init}}
    %{\tcb{\ltr}{\refval{q}{\gs}}{\tyref{\lt'}{q}{\gt}}{mq}{\lt}}

  \and
  \infer[HT-REFMUT]
    {(\lt, \lt') \in \ltr \\ \tcb{\ltr}{\gs}{\gt}{mq}{\lt} \\ \gs\ \textrm{init}}
    {\tcb{\ltr}{\refval{\qmut}{\gs}}{\tyref{\lt'}{\qmut}{\gt}}{mq}{\lt}}
  \and
  \infer[HT-REFIMM]
    {(\lt, \lt') \in \ltr \\ \tcb{\ltr}{\gs}{\gt}{\qimm}{\lt} \\ \gs\ \textrm{init}}
    {\tcb{\ltr}{\refval{\qimm}{\gs}}{\tyref{\lt'}{\qimm}{\gt}}{mq}{\lt}}


\end{mathpar}

where $\gs\ \textrm{init}$ is a judgment that ensures a shadow is fully initialized,
i.e.\ it cannot reach a $\uninit$.
Note that the interior of immutable borrowed references is always controlled
by an immutable loan. This prevents mutable loans of the interior of immutable borrows,
which are nonsensical.

%need to think about how the qualifier of a borrowed reference impacts the kind of controlling loan of its contents. Ex: a mutable loan of the dereference of an immutable borrow should be forbidden. Dereferences of mutable borrows can be loaned either way...
%Perhaps the interior of an immutable borrow should be controlled by $\qimm$ always,
%and the interior of a mutable borrow should be controlled by the incoming controller.
%That seems sensible since immutable loans overpower mutable loans ... it's a similar pattern.

\subsubsection*{Bounded Shadow Heap Well-Formedness}
With our shadow typing updated to check banks,
we can update what we mean by a well-formed shadow heap.
The lifetime relation and the lifetime map are now included.
We ensure the lifetime map is well-formed by requiring its domain to be
the same as the domain of the type context and the shadow heap, and by requiring
that every lifetime in its range to be in the lifetime relation.

\begin{mathpar}
\mprset{flushleft}
\infer{
\dom \gG = \dom \gU = \dom L \\\\
\forall x \in \dom \gG.~\gG(x)~\textrm{closed} \\\\
\forall x \in \dom L.~(L(x),L(x)) \in L \\\\
\forall x \in \dom \gU.~\tcb{\ltr}{\gU(x)}{\gG(x)}{\mqnone}{L(x)}
}
{\tb{\ltr}{\gU}{\gG}{L}}
\end{mathpar}

\subsection*{Ensuring Operations Do Not Violate Loans}
The most important duty of the borrow checker is making sure that
performing some operation does not violate the promises of existing loans.
We cannot read a path that has promised unique access to another.
We cannot write to a path that has promised to remain unchanged or that we cannot read.
We cannot move from a path that we do not own or that we cannot write.

Additionally, we can only use a path that is fully initialized, or our program might
try to touch a uninitialized value and get stuck.
This means that $\shallow{\gU}{\lv{x}{p}}{\gs}$ holds and that $\gs\ \textrm{init}$ holds. 
That is, enough memory is initialized to evaluate the L-value to a memory location,
and everything reachable from that location is initialized.
We use these dual judgments as a base to define our validity 
checks for reading, writing, and moving.
Reading will be more restricted than simple full initialization.
Writing will be more restricted than reading.
Moving will be more restricted than writing.

\subsubsection*{Reading}
The restriction we add to full initialization to verify that reading a path is valid is
that no part of the path or anything reachable from the path has been mutable loaned.
A mutable loan promises unique access via the corresponding mutable borrowed reference, 
which precludes reading the loaned path directly.
We can also short-circuit our check in a two cases.
If the path we want to read dereferences an immutable borrow, then we do not need
to check the interior of the borrow. By its nature, the immutable borrow guarantees readability,
and any mutable loaning of the interior is already forbidden.
For similar reasons, when checking the memory reachable from the path 
we do not need to examine the interior of immutable borrows.
\newline

\fbox{$\cnml{\gs}$}

\begin{mathpar}
  \infer[CNML-INT]{\nm{\$}}{\cnml{\$:\tyint}} \and
  \infer[CNML-OWN]{\nm{\$} \\ \cnml{\gs}}{\cnml{\$:\own{\gs}}} \and
  \infer[CNML-REFIMM]{\nm{\$}}{\cnml{\$:\refval{\qimm}{\gs}}} \and
  \infer[CNML-REFMUT]{\nm{\$} \\ \cnml{\gs}}{\cnml{\$:\refval{\qmut}{\gs}}}
\end{mathpar}

\fbox{$\canread{p}{\gs}$}

\begin{mathpar}
  \infer[CR-BASE] {\cnml{\gs}} {\canread{\base}{\gs}} \and
  \infer[CR-DEOWN]
    {\nm{\$} \\ \canread{p}{\gs} }
    {\canread{\deref{p}}{\$:\own{\gs}}} \and
  \infer[CR-DEREFIMM]
    {\nm{\$}}
    {\canread{\deref{p}}{\$:\refval{\qimm}{\gs}}} \and
  \infer[CR-DEREFMUT]
    {\nm{\$} \\ \canread{p}{\gs}}
    {\canread{\deref{p}}{\$:\refval{\qmut}{\gs}}}
\end{mathpar}

where one would check that $\lv{x}{p}$ is readable by $\canread{p}{\gU(x)}$.

\subsubsection*{Writing}
In order to safely write a path, no part of the path or anything reachable from the path
can be loaned at all. An immutable loan implies the data is frozen, 
which prevents us from writing. A mutable loan implies something else has unique acces
to the data, which also prevents us from writing.
Additionally, we cannot write to the interior of an immutable borrow, even if we can
write to the borrowed reference itself. Memory reachable from that immutable borrow
is frozen.
\newline

\fbox{$\cnl{\gs}$}

\begin{mathpar}
  \infer[CNL-INT]{ }{\cnl{\emptyset : \tyint}} \and
  \infer[CNL-OWN]{\cnl{\gs}}{\cnl{\emptyset : \own{\gs}}} \and
  \infer[CNL-REF]{\cnl{\gs}}{\cnl{\emptyset : \refval{q}{\gs}}}
\end{mathpar}

Note that the borrowed reference cases of have merged again after being separate for reading.
We cannot guarantee that the interior of an immutable borrow will not be loaned out without
actually checking. However, note that $\cnl{\gs}$ implies $\cnml{\gs}$.
\newline

\fbox{$\cw{p}{\gs}$}

\begin{mathpar}
  \infer[CW-BASE]{\cnl{\gs}}{\cw{\base}{\gs}} \and
  \infer[CW-DEOWN]{\cw{p}{\gs}}{\cw{\deref{p}}{\emptyset : \own{\gs}}} \and
  \infer[CW-DEREFMUT]{\cw{p}{\gs}}{\cw{\deref{p}}{\emptyset : \refval{\qmut}{\gs}}}
\end{mathpar}

Again, note that $\cw{p}{\gs}$ implies $\canread{p}{\gs}$.
Rust requires that the ability to write implies the ability to read, which is
reflected in our judgments.

\subsubsection*{Moving}
The final operation, moving out of a path, requires ownership over the path.
Moving is used when the value being read has an affine type, which prevents it 
from being merely copied. The old location of the value must be purged.
This requires the ability to write, but since it leaves the original path uninitialized
it cannot be used via mutable borrows. Doing so would violate the guarantee that borrowed
references always point to initialized memory.
\newline

\fbox{$\cm{p}{\gs}$}

\begin{mathpar}
  \infer[CM-BASE]{\cnl{\gs}}{\cm{\base}{\gs}} \and
  \infer[CM-DEOWN]{\cm{p}{\gs}}{\cm{\deref{p}}{\emptyset : \own{\gs}}}
\end{mathpar}

Note that $\cm{p}{\gs}$ implies $\cw{p}{\gs}$.

\subsection*{Affine Types and Copyability}
Values of an affine type cannot be used more than once, e.g.\ unique pointers.
We model this in Patina by deinitializing an affine value on use.
However, we do not want to do this unnecessarily so we need to distinguish between
affine and non-affine types. Affine types must be moved, 
but non-affine types can be merely copied.
In Patina, unique pointers and borrowed mutable references are the sources of affineness.
Unique pointers must be affine to preserve their uniqueness.
Borrowed mutable references must be affine to preserve the unique access they promise.
\newline

\fbox{$\tycopy{\gt}$}

\begin{mathpar}
  \infer[C-INT]{ }{\tycopy{\tyint}} \and
  \infer[C-REFIMM]{ }{\tycopy{\tyref{\lt}{\qimm}{\gt}}}
\end{mathpar}

\fbox{$\tymove{\gt}$}

\begin{mathpar}
  \infer[M-OWN]{ }{\tymove{\own{\gt}}} \and
  \infer[M-REFMUT]{ }{\tymove{\tyref{\lt}{\qmut}{\gt}}}
\end{mathpar}

The two judgments here are not particularly interesting without compound types,
but they still create a necessary distinction.
Note that these two judgments partition the set of types.

\subsection*{Using a L-value}
Now that we can check that an operation is safe and we can determine from the type
which operation to perform, we can specify how to check that using the value at a L-value 
is safe and track how doing so changes the shadow heap.
\newline

\fbox{$\uselv{\gG}{\gU}{\lv{x}{p}}{\gt}{\gU'}$}

\begin{mathpar}
  \infer[UP-COPY]
    {\tc{\gG}{\lv{x}{p}}{\gt} \\ \tycopy{\gt} \\ \canread{p}{\gU(x)}}
    {\uselv{\gG}{\gU}{\lv{x}{p}}{\gt}{\gU}}
  \and
  \infer[UP-MOVE]
    {\tc{\gG}{\lv{x}{p}}{\gt} \\ \tymove{\gt} \\ \cm{p}{\gU(x)} 
    \\ \textsc{use}(\gU,\lv{x}{p},\gU') }
    {\uselv{\gG}{\gU}{\lv{x}{p}}{\gt}{\gU'}}
\end{mathpar}

where $\textsc{use}(\gU,\lv{x}{p},\gU')$ modifies the shadow 
of $\lv{x}{p}$ to deinitialize the affine values.

\begin{conj}[L-Value Use Preservation]
  If $\vdash\ltr$, $\tb{\ltr}{\gU}{\gG}{L}$, and $\uselv{\gG}{\gU}{\lv{x}{p}}{\gt}{\gU'}$
  then $\tb{\ltr}{\gU'}{\gG}{L}$.
\end{conj}

\begin{proof}[Justification]
  Using a path either leaves the shadow heap unchanged or deinitializes only a part of it,
  neither of which should change the types of shadows or the lifetimes of loans.
  Deinitializing part of the shadow heap could invalidate existing loans,
  but the \text{can-read} and \textsc{can-move} checks should rule out those cases.
\end{proof}

\subsection*{Ensuring Loan Promises Can Be Upheld}
When creating a borrowed reference, the borrow checker must ensure
that is it actually possible to make the promises that the loan demands.
For example, we cannot promise the interior of a borrowed reference will be 
valid for longer than that reference's lifetime, nor can we promise that
the interior of a immutable borrowed reference will have unique access.

\subsubsection*{Validity}
Both kinds of borrowed references require that their referent be initialized for
the lifetime of the reference, which here we call \emph{validity}.
Variables are valid for their entire lifetime and can be loaned out for any duration
not longer than that.
Unique pointers inherit the lifetime of their owner.
Borrowed references specify how long they are valid by their lifetime, and they
can be loaned out for any duration not longer than that. Since we already know
the interior of a borrowed reference is valid for the lifetime of that reference,
we do not need to examine further.
\newline

\fbox{$\valid{\gG}{\ltr}{L}{\lv{x}{p}}{\lt}$}

\begin{mathpar}
  \infer[VF-BASE]{(\lt,L(x))\in\ltr}{\valid{\gG}{\ltr}{L}{\lv{x}{\base}}{\lt}} \and
  \infer[VF-DEOWN]
    {\tc{\gG}{\lv{x}{p}}{\own{\gt}} \\ \valid{\gG}{\ltr}{L}{\lv{x}{p}}{\lt}}
    {\valid{\gG}{\ltr}{L}{\lv{x}{\deref{p}}}{\lt}} \and
  \infer[VF-DEREF]
    {\tc{\gG}{\lv{x}{p}}{\tyref{\lt'}{q}{\gt}} \\ (\lt,\lt')\in\ltr}
    {\valid{\gG}{\ltr}{L}{\lv{x}{\deref{p}}}{\lt}}
\end{mathpar}

\subsubsection*{Uniqueness}
Mutable loans require the loaned memory have \emph{unique} access.
If the path being loaned has unique access, then it can give that access to the 
new mutable borrow.
Variables, being top level, are inherently unique along with unique pointers.
Mutable borrows provide unique access to their interiors for their duration.
Immutable borrows are inherently not unique.
\newline

\fbox{$\unique{\gG}{\ltr}{\lv{x}{p}}{\lt}$}

\begin{mathpar}
  \infer[UF-BASE]{ }{\unique{\gG}{\ltr}{\lv{x}{\base}}{\lt}} \and
  \infer[UF-DEOWN]
    {\tc{\gG}{\lv{x}{p}}{\own{\gt}} \\ \unique{\gG}{\ltr}{\lv{x}{p}}{\lt}}
    {\unique{\gG}{\ltr}{\lv{x}{\deref{p}}}{\lt}} \and
  \infer[UF-DEREFMUT]
    {\tc{\gG}{\lv{x}{p}}{\tyref{\lt'}{\qmut}{\gt}} \\ (\lt,\lt')\in\ltr
    \\ \unique{\gG}{\ltr}{\lv{x}{p}}{\lt}}
    {\unique{\gG}{\ltr}{\lv{x}{\deref{p}}}{\lt}}
\end{mathpar}

\subsubsection*{Freezability}
Immutable loans require the loaned memory to be \emph{frozen}, 
unchangable by anything, for the duration of the loan.
When creating an immutable borrow, we need to check a slightly weaker property:
\emph{freezable}, unchangable by anything \emph{else}.
A freezable path can be mutated, but only by the path itself.
If the path does not mutate itself, e.g.\ by loaning itself immutably, then we know it is frozen.
Unique paths are inherently freezable since no other path exists.
Dereferencing through an immutable borrow, while not unique, is freezable because
the interior of the borrow is already guaranteed to be frozen for its lifetime.
\newline

\fbox{$\freezable{\gG}{\ltr}{\lv{x}{p}}{\lt}$}

\begin{mathpar}
  \infer[FF-BASE]{ }{\freezable{\gG}{\ltr}{\lv{x}{\base}}{\lt}} \and
  \infer[FF-DEOWN]
    {\tc{\gG}{\lv{x}{p}}{\own{\gt}} \\ \freezable{\gG}{\ltr}{\lv{x}{p}}{\lt}}
    {\freezable{\gG}{\ltr}{\lv{x}{\deref{p}}}{\lt}} \and
  \infer[FF-DEREFMUT]
    {\tc{\gG}{\lv{x}{p}}{\tyref{\lt'}{\qmut}{\gt}} \\ (\lt,\lt')\in\ltr 
    \\ \freezable{\gG}{\ltr}{\lv{x}{p}}{\lt}}
    {\freezable{\gG}{\ltr}{\lv{x}{\deref{p}}}{\lt}} \and
  \infer[FF-DEREFIMM]
    {\tc{\gG}{\lv{x}{p}}{\tyref{\lt'}{\qimm}{\gt}} \\ (\lt,\lt')\in\ltr }
    {\freezable{\gG}{\ltr}{\lv{x}{\deref{p}}}{\lt}}
\end{mathpar}

Note how $\unique{\gG}{\ltr}{\lv{x}{p}}{\lt}$ implies $\freezable{\gG}{\ltr}{\lv{x}{p}}{\lt}$.
This implies that anything we can borrow mutably we can also borrow immutably,
which accurately reflects Rust.

\subsection*{Expression Typing}

\subsection*{Using a Layout}

\subsection*{Expression Evaluation}

\subsection*{Expression Preservation}

\subsection*{Expression Progress}

\section*{Statements}
\subsection*{Statement Typing}
\subsection*{Statement Evaluation}
\subsection*{Statement Preservation}
\subsection*{Statement Progress}
\section*{Patina Soundness}


\end{document}
