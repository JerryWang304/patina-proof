%\section*{Introduction}
%Rust claims to ensure memory safety without a garbage collector.
%We would like to verify this claim.

%Rust ensures safety by combining affine types with checked borrowed references.
%Affine types ensure that every piece of memory has a unqiue owner responsible for deallocation.
%Borrowed references allow limited aliasing that avoids 
%aliased, mutable memory and dangling references
%Rust's usage of affine types for memory management is standard except for one addition.
%Rust allows consumed affine cells that are still in scope to be re-initialized and used again.
%Rust's borrowed references utilize a region system very similar to Cyclone's.
%The primary difference is in the details of the borrow checker.
%Our main goal is to prove that these differences do not introduce unsoundness.

%\section*{Patina}
%We work with a small core model of Rust called Patina, which focuses
%on the memory management aspects of Rust.
%Unlike Rust, Patina uses explicit heap deallocation and does not support closures.
%Patina has three layers: L-values, R-values, and statements.
%We will discuss them in order.

\section*{Overview}
Rust makes strong claims about ensuring memory safety without garbage collection.
We would like to prove that those claims are true.
To that end, we use a small model of Rust, called Patina, that characterizes the key
features of Rust under the most suspicion, namely its memory management.

Rust memory management has two goals:
no memory should ever be orphaned and no uninitialized memory should ever be read.
Rust tries to achieve this by maintaining two invariants:
all memory has a unique owner responsible for deallocating it and
no memory is ever simultaneously aliased and mutable.
These invariants simplify the situation enough so that
Rust only needs to track which L-values are uninitialized and which have been borrowed.
Ownership tracking of heap memory is managed by unique pointers.
Safe aliasing is managed by borrowed references.

The Patina model has three layers.
The innermost layer deals with L-values and ensures no uninitialized pointer
ever needs to be dereferenced. It also ensures that initialization data correctly
models the runtime memory.
The middle layer deals with R-values.
It ensures that an L-values used do not violate initialization or borrow restrictions.
It also ensures that borrowed references are not created unless they are safe.
The outer layer deals with statements.
It mostly just chains together the guarantees of the L-value and R-value layers.
However, it is also responsible for ensuring no orphans would be created when deallocating memory.

\subsection*{Unique Pointers in Rust}
In Rust, unique pointers are the owners of heap memory.
Heap memory is allocated when a new unique pointer is created.
Heap memory is freed when a unique pointer falls out of scope.
\begin{verbatim}
{
  let x: ~int; // A stack variable containing a unique pointer to an integer
  x = ~3; // Allocates a heap integer, initializes it with 3, and stores the pointer in x
  // The heap memory owned by x is freed when it falls out of scope
}
\end{verbatim}

To avoid double frees, unique pointers must remain unique.
Thus, when a unqiue pointer would be copied the original must be rendered unsuable.
That is, they are moved rather than copied.
\begin{verbatim}
{
  let x: ~int = ~3;
  let y: ~int = x; // x is moved into y. x is no longer usable.
  let z: ~int = x; // ERROR!
}
\end{verbatim}

However, these deinitialized paths can be reinitialized.
\begin{verbatim}
{
  let x: ~int = ~3;
  let y: ~int = x; // x is now deinitialized
  x = ~1; // x is initialized again
}
\end{verbatim}

Unique pointers can also be freed if they would be orphaned by assignment.
\begin{verbatim}
{
  let x: ~~int = ~~3;
  *x = ~1; // The ~3 that *x points too would be orphaned here. It is freed instead.
}
\end{verbatim}

This is not a dynamic check. 
The compiler detects when heap memory should be freed and inserts the necessary code.

\subsection*{Unique Pointers in Patina}
To check that the Rust compiler does this correctly, Patina uses a shallow free statement.
The only way to deallocate a unique pointer in Patina is with these explicit frees.
\begin{verbatim}
{
  let x: ~~int = ~~3;
  //free(x); // ERROR! would orphan *x
  // The proper way
  free(*x);
  free(x);
}
\end{verbatim}
However, the free statement is only valid for initialized pointers, which prevents double frees.
\begin{verbatim}
{
  let x: ~int = ~3;
  free(x); // *x is now deallocated. x is uninitialized
  free(x); // ERROR! x is not initialized
}
\end{verbatim}
Finally, the free statement requires that the pointer is unaliased, preventing dangling pointers.
\begin{verbatim}
{
  let x: ~int = ~3;
  let y: &int = &*x; // Create an immutable borrowed reference to *x
  free(x); // ERROR! would make y a dangling pointer
}
\end{verbatim}

\subsection*{Borrowed References}
Borrowed references are a means to provide safe, temporary aliases in Rust.
They do so by enforcing two invariants.
The first is that references are always valid.
%They do so by enforcing two invariants: 
%references are always valid and
%no memory is ever simultaneously aliased and mutable.
%As with unique pointers, these are static checks.
\begin{verbatim}
{
  let x: int; // An uninitialized stack integer
  let y: &int = &x; // ERROR! *y would point to uninitialized memory
}
\end{verbatim}
This also extends to cases where the referent falls out of scope too soon.
\begin{verbatim}
{
  let y: &int;
  {
    let x: int = 3;
    y = &x; // ERROR! x does not live long enough
  } // x would be freed here
  // *y would point to unallocated memory here
}
\end{verbatim}
The second invariant is to prevent any memory from being simultaneously aliased and mutable.
This is to avoid situtations like the following:
\begin{verbatim}
{
  let x: Option<int> = Some(3);
  let y: &int = match x {
    Some(ref z) => z,
    None => fail!(), // Impossible
  }
  // If x is mutable, then the payload 3 is both aliased and mutable
  x = None // The payload is now uninitialized so *y is a dangling pointer
}
\end{verbatim}
Therefore immutable borrows freeze the underlying data.
It cannot be mutated by anything while the borrow exists.
However, it is freely copyable.
\begin{verbatim}
{
  let x: int = 3;
  let y: &int = &x;
  let z: &int = y; // Both y and z are readable
  x = 2; // ERROR! x is frozen by y
}
\end{verbatim}
Conversely, mutable borrows forbid all other access to the underlying data.
It does allow the data to be mutated, but it is an affine datatype like unique pointers.
\begin{verbatim}
{
  let x: int = 3;
  let y: &mut int = &mut x;
  //let z: int = x; // ERROR! cannot use x while mutable borrowed
  let z: int = *y; // OK

  let w: &mut int = y; // Moves y into w. y is not uninitialized
  z = *y; // ERROR! y is uninitialized
}
\end{verbatim}
As with unique pointers, these checks are all static.

\section*{Types}
Values in Patina can be described by the following types:
\[
\begin{array}{lccl}
\textrm{Lifetime} & \lt & & \\
\textrm{Type Variable} & X & & \\
\textrm{Qualifier} & q & \bnfdef & \qimm \bnfalt \qmut \\
\textrm{Type} & \gt & \bnfdef & \tyint \bnfalt \own{\gt} \bnfalt \tyref{\lt}{q}{\gt} \bnfalt 
				\subvar{\gt}{i}{n} \bnfalt \subrec{\gt}{i}{n} \bnfalt 
				\tyfix{X}{\gt} \bnfalt X \\
\end{array}
\]

$\own{\gt}$ is a unique pointer to a $\gt$.
$\tyref{\lt}{q}{\gt}$ is a borrowed reference to a $\gt$
providing mutability guarantee $q$ for lifetime $\lt$.
$\subvar{\gt}{i}{n}$ is a variant type.
$\subrec{\gt}{i}{n}$ is a tuple type.
Finally, $\tyfix{X}{\gt}$ is a recursive type.
